\documentclass[11pt, a4paper, titlepage]{report}
\usepackage{amsmath}
\usepackage{natbib}
\usepackage{a4wide}
\usepackage{url}
\usepackage{multirow}
\usepackage{amsfonts}
\newcommand{\release}{3.3.0}
\newcommand{\JAGS}{\textsf{JAGS}}
\newcommand{\rjags}{\textsf{rjags}}
\newcommand{\BUGS}{\textsf{BUGS}}
\newcommand{\OpenBUGS}{\textsf{OpenBUGS}}
\newcommand{\R}{\textsf{R}}
\newcommand{\CODA}{\textsf{coda}}
\begin{document}

\title{JAGS Version \release\ user manual}
\author{Martyn Plummer}
\date{1 October 2012}
\maketitle

\tableofcontents

\chapter{Introduction}

JAGS is Just Another Gibbs Sampler.  It is a program for the analysis
of Bayesian models using Markov Chain Monte Carlo (MCMC) which is not
wholly unlike
\OpenBUGS\ (\url{http://www.openbugs.info}). \JAGS\ was written
with three aims in mind: to have an engine for the \BUGS\ language
that runs on Unix; to be extensible, allowing users to write their own
functions, distributions, and samplers; and to be a platform for
experimentation with ideas in Bayesian modelling.

\JAGS\ is designed to work closely with the \R\ language and
environment for statistical computation and graphics
(\url{http://www.r-project.org}).  You will find it useful to install
the \CODA\ package for \R\ to analyze the output. You can also use the
\rjags\ package to work directly with \JAGS\ from within R (but note
that the \rjags\ package is not described in this manual).

\JAGS\ is licensed under the GNU General Public License
version 2. You may freely modify and redistribute it under certain
conditions (see the file \texttt{COPYING} for details).

\chapter{Running a model in \JAGS}

\JAGS\ is designed for inference on Bayesian models using Markov Chain
Monte Carlo (MCMC) simulation.  Running a model refers to generating
samples from the posterior distribution of the model parameters.  This
takes place in five steps:
\begin{enumerate}
\item Definition of the model
\item Compilation
\item Initialization
\item Adaptation and burn-in
\item Monitoring
\end{enumerate}
The next stages of analysis are done outside of \JAGS: convergence
diagnostics, model criticism, and summarizing the samples must be done
using other packages more suited to this task. There are several
\R\ packages designed for analyzing MCMC output, and \JAGS\ can be
used from within \R\ using the \rjags\ package.

\section{Definition}

There are two parts to the definition of a model in \JAGS: a
description of the model and the definition of the data.

\subsection{Model definition}

The model is defined in a text file using a dialect of the
\BUGS\ language.  The model definition consists of a series of
relations inside a block delimited by curly brackets \verb+{+ and
\verb+}+ and preceded by the keyword \verb+model+. Here is the standard
linear regression example:

\begin{verbatim}
model {
    for (i in 1:N) {
          Y[i]   ~ dnorm(mu[i], tau)
          mu[i] <- alpha + beta * (x[i] - x.bar)
    }
    x.bar   <- mean(x)
    alpha    ~ dnorm(0.0, 1.0E-4)
    beta     ~ dnorm(0.0, 1.0E-4)
    sigma   <- 1.0/sqrt(tau)
    tau      ~ dgamma(1.0E-3, 1.0E-3)
}
\end{verbatim}

Each relation defines a node in the model in terms of other nodes that
appear on the right hand side. These are referred to as the parent
nodes. Taken together, the nodes in the model (together with the
parent/child relationships represented as directed edges) form a
directed acyclic graph. The very top-level nodes in the graph, with no
parents, are constant nodes, which are defined either in the model
definition ({\em e.g.}  \verb+1.0E-3+), or in the data file ({\em
  e.g.}  \verb+x[1]+).

Relations can be of two types. A {\em stochastic relation} (\verb+~+)
defines a stochastic node, representing a random variable in the
model. A {\em deterministic relation} (\verb+<-+) defines a
deterministic node, the value of which is determined exactly by the
values of its parents.

Nodes defined by a relation are embedded in named arrays. Array names may
contain letters, numbers, decimal points and underscores, but they must
start with a letter.  The node array \verb+mu+ is a vector of length
$N$ containing $N$ nodes (\verb+mu[1]+, $\ldots$, \verb+mu[N]+). The
node array \verb+alpha+ is a scalar.  \JAGS\ follows the S language
convention that scalars are considered as vectors of length 1. Hence
the array \verb+alpha+ contains a single node \verb+alpha[1]+.

Deterministic nodes do not need to be embedded in node arrays. The
node \verb+Y[i]+ could equivalently be defined as
\begin{verbatim}
Y[i] ~ dnorm(alpha + beta * (x[i] - x.bar), tau)
\end{verbatim}
In this version of the model definition, the node previously defined
as \verb+mu[i]+ still exists, but is not accessible to the user as it
does not have a name.  This ability to hide deterministic nodes by
embedding them in other expressions underscores an important fact:
only the stochastic nodes in a model are really
important. Deterministic nodes are merely a syntactically convenient
way of describing the relations between stochastic nodes, or
transformations of them.

\subsection{Data}
\label{section:data}

The data are defined in a separate file from the model definition, in
the format created by the \texttt{dump()} function in \R (See appendix
\cite{appendix:data}).  Data values may be supplied for stochastic
nodes and constants (including constant values used inside for
loops). It is an error to supply a data value for a deterministic
node. (See, however, section \ref{section:obfun} on observable
functions).

Here are the data for the \verb+LINE+ example:
\begin{verbatim}
`x` <- c(1, 2, 3, 4, 5)
#R-style comments, like this one, can be embedded in the data file
`Y` <- c(1, 3, 3, 3, 5)
`N` <- 5
\end{verbatim}

The unobserved stochastic nodes are referred to as the {\em
  parameters} of the model. The data file therefore defines the
parameters of the model by omission. In the \verb+LINE+ example, the
parameters are \texttt{alpha}, \texttt{beta} and \texttt{tau}.

If a node array contains both observed and unobserved nodes, then the
data should contain missing values (\texttt{NA}) for the unobserved
elements. In the \verb+LINE+ example, if \verb+Y[2]+ and \verb+Y[5]+
were unobserved, then the data would be
\begin{verbatim}
`Y` <- c(1, NA, 3, 3, NA)
\end{verbatim}
Multivariate nodes cannot be partially observed, so if a node takes up
two or more elements of a node array, then the corresponding data
values must be all present or all missing.

\subsection{Node Array dimensions}

\subsubsection*{Array declarations}

\JAGS\ allows the option of declaring the dimensions of node arrays in
the model file. The declarations consist of the keyword \texttt{var}
(for variable) followed by a comma-separated list of array names, with
their dimensions in square brackets. The dimensions may be given in
terms of any expression of the data that returns a single integer
value.

In the linear regression example, the model block could be preceded by
\begin{verbatim}
var x[N], Y[N], mu[N], alpha, beta, tau, sigma, x.bar;
\end{verbatim}

\subsubsection*{Undeclared nodes}

If a node array is not declared then JAGS has three methods of
determining its size.
\begin{enumerate}
\item {\bf Using the data.}  The dimension of an undeclared node array
  may be inferred if it is supplied in the data file.
\item {\bf Using the left hand side of the relations.}  The maximal
  index values on the left hand side of a relation are taken to be the
  dimensions of the node array.  For example, in this case:
\begin{verbatim}
for (i in 1:N) {
   for (j in 1:M) {
      Y[i,j] ~ dnorm(mu[i,j], tau)
   }
}
\end{verbatim}
$Y$ would be inferred to be an $N \times M$ matrix. This method cannot 
be used when there are empty indices ({\em e.g.} \verb+Y[i,]+) on the left
hand side of the relation.
\item {\bf Using the dimensions of the parents} If a whole node array
  appears on the left hand side of a relation, then its dimensions can
  be inferred from the dimensions of the nodes on the right hand side.
  For example, if \verb+A+ is known to be an $N \times N$ matrix
  and
\begin{verbatim}
B <- inverse(A)
\end{verbatim}
Then \verb+B+ is also an $N \times N$ matrix.
\end{enumerate}

\subsubsection*{Querying array dimensions}  

The \JAGS\ compiler has two built-in functions for querying array
sizes.  The \verb+length()+ function returns the number of elements in
a node array, and the \verb+dim()+ function returns a vector
containing the dimensions of an array.  These two functions may be
used to simplify the data preparation. For example, if \verb+Y+
represents a vector of observed values, then using the \verb+length()+
function in a for loop:
\begin{verbatim}
for (i in 1:length(Y)) {
    Y[i] ~ dnorm(mu[i], tau)
}
\end{verbatim}
avoids the need to put a separate data value \verb+N+ in the file
representing the length of \verb+Y+.  

For multi-dimensional arrays, the \verb+dim+ function serves a similar
purpose. The \verb+dim+ function returns a vector, which must be stored
in an array before its elements can be accessed. For this reason, calls
to the \verb+dim+ function must always be in a data block (see section
\ref{section:data:tranformations}).
\begin{verbatim}
data {
   D <- dim(Z)
}
model {
   for (i in 1:D[1]) {
      for (j in 1:D[2]) {
         Z[i,j] <- dnorm(alpha[i] + beta[j], tau)
      }
   }
   ...
}
\end{verbatim}
Clearly, the \verb+length()+ and \verb+dim()+ functions can only
work if the size of the node array can be inferred, using one of the
three methods outlined above.

Note: the \verb+length()+ and \verb+dim()+ functions are different
from all other functions in \JAGS: they do not act on nodes, but only
on node {\em arrays}. As a consequence, an expression such as
\verb+dim(a %*% b)+ is syntactically incorrect.

\section{Compilation}

When a model is compiled, a graph representing the model is created in
computer memory. Compilation can fail for a number of reasons:
\begin{enumerate}
\item The graph contains a directed cycle.  These are forbidden
in \JAGS.
\item A top-level parameter is undefined. Any node that is used on
the right hand side of a relation, but is not defined on the left
hand side of any relation, is assumed to be a constant node. Its value
must be supplied in the data file. 
\item The model uses a function or distribution that has not been
defined in any of the loaded modules.
\end{enumerate}
The number of parallel chains to be run by \JAGS\ is also defined at
compilation time.  Each parallel chain should produce an independent
sequence of samples from the posterior distribution. By default,
\JAGS\ only runs a single chain.

\section{Initialization}

Before a model can be run, it must be initialized. There are three
steps in the initialization of a model:
\begin{enumerate}
\item The initial values of the model parameters are set.
\item A Random Number Generator (RNG) is chosen for each parallel chain,
  and its seed is set.
\item The Samplers are chosen for each parameter in the model. 
\end{enumerate}

\subsection{Parameter values}

The user may supply initial value files -- one for each chain --
containing initial values for the model parameters. The files may not
contain values for logical or constant nodes. The format is the same
as the data file (see (See appendix \cite{appendix:data}).
Section \label{parameters:in} describes how to read the initial value
files into the model.

As with the data file, you may supply missing values in the initial
values file.  This need typically arises with contrast parameters.
Suppose $X$ is a categorical covariate taking values from 1 to
4. There is one parameter for each level of x ($\beta_1 \ldots
\beta_4$) , but the first parameter $\beta_1$ is set to zero for
identifiability. The remaining parameters $\beta_2 \ldots \beta_4$
represent contrasts with respect to the first level of $X$.
\begin{verbatim}
for (i in 1:N) {
   Y[i] ~ alpha + beta[x[i]]
}
# Prior distribution
alpha ~ dnorm(0, 1.0E-3)
beta[1] <- 0
for(i in 2:4) {
   beta[i] ~ dnorm(0, 1.0E-3)
}
\end{verbatim}
A suitable initial value for \verb+beta+ would be
\begin{verbatim}
`beta` <- c(NA, 1.03, -2.03, 0.52)
\end{verbatim}
This allows parameter values to be supplied for the stochastic
elements of \verb+beta+ but not the constant first element.

If initial values are not supplied by the user, then each parameter
chooses its own initial value based on the values of its parents.  The
initial value is chosen to be a ``typical value'' from the prior
distribution. The exact meaning of ``typical value'' depends on the
distribution of the stochastic node, but is usually the mean, median,
or mode.

If you rely on automatic initial value generation and are running
multiple parallel chains, then the initial values will be the same in
all chains.  You may not want this behaviour, especially if you are
using the Gelman and Rubin convergence diagnostic, which assumes that
the initial values are over-dispersed with respect to the posterior
distribution. In this case, you are advised to set the starting values
manually using the "parameters in" statement.

\subsection{RNGs}
\label{section:rngs}

Each chain in \JAGS\ has its own random number generator (RNG). RNGs
are more correctly referred to as {\em pseudo}-random number
generators. They generate a sequence of numbers that merely looks
random but is, in fact, entirely determined by the initial state.  You
may optionally set the name of the RNG and its initial state in the
initial values file.

The name of the RNG is set as follows. 
\begin{verbatim}
.RNG.name <- "name"
\end{verbatim}
There are four RNGs supplied by the \texttt{base} module in \JAGS\
with the following names:
\begin{verbatim}
"base::Wichmann-Hill"
"base::Marsaglia-Multicarry"
"base::Super-Duper"
"base::Mersenne-Twister"
\end{verbatim}

There are two ways to set the starting state of the RNG. The simplest
is to supply an integer value to \texttt{.RNG.seed}, {\em e.g.}
\begin{verbatim}
".RNG.seed" <- 314159
\end{verbatim}
The second is way to save the state of the RNG from one JAGS session
(see the ``PARAMETERS TO'' statement, section \ref{parameters:to}) and
use this as the initial state of a new chain. The state of any RNG in
JAGS can be saved and loaded as an integer vector with the name
\texttt{.RNG.state}. For example,
\begin{verbatim}
".RNG.state" <- as.integer(c(20899,10892,29018))
\end{verbatim}
is a valid state for the Marsaglia-Multicarry generator.  You cannot
supply an arbitrary integer to \texttt{.RNG.state}. Both the length of
the vector and the permitted values of its elements are determined by
the class of the RNG. The only safe way to use \texttt{.RNG.state} is
to re-use a previously saved state.

If no RNG names are supplied, then RNGs will be chosen automatically
so that each chain has its own independent random number stream.  The
exact behaviour depends on which modules are loaded. The \texttt{base}
module uses the four generators listed above for the first four
chains, then recycles them with different seeds for the next four
chains, and so on.  

By default, \JAGS\ bases the initial state on the time stamp. This
means that, when a model is re-run, it generates an independent set of
samples. If you want your model run to be reproducible, you must
explicitly set the \verb+.RNG.seed+ for each chain.

\subsection{Samplers}

A Sampler is an object that acts on a set of parameters and updates
them from one iteration to the next. During initialization of the
model, Samplers are chosen automatically for all parameters. 

The Model holds an internal list of {\em Sampler Factory} objects,
which inspect the graph, recognize sets of parameters that can be
updated with specific methods, and generate Sampler objects for
them. The list of Sampler Factories is traversed in order, starting with
sampling methods that are efficient, but limited to certain specific
model structures and ending with the most generic, possibly
inefficient, methods. If no suitable Sampler can be generated for one
of the model parameters, an error message is generated.

The user has no direct control over the process of choosing
Samplers. However, you may indirectly control the process by loading a
module that defines a new Sampler Factory. The module will insert the
new Sampler Factory at the beginning of the list, where it will be
queried before all of the other Sampler Factories. You can also 
optionally turn on and off sampler factories using the ``SET FACTORY''
command. See \ref{set:factory}.

A report on the samplers chosen by the model, and the stochastic nodes
they act on, can be generated using the ``SAMPLERS TO'' command. See 
section \ref{samplers:to}.
 
\section{Adaptation and burn-in}

In theory, output from an MCMC sampler converges to the target
distribution ({\em i.e.} the posterior distribution of the model
parameters) in the limit as the number of iterations tends to
infinity. In practice, all MCMC runs are finite.  By convention, the
MCMC output is divided into two parts: an initial ``burn-in'' period,
which is discarded, and the remainder of the run, in which the output
is considered to have converged (sufficiently close) to the target
distribution. Samples from the second part are used to create
approximate summary statistics for the target distribution.

By default, \JAGS\ keeps only the current value of each node in the
model, unless a monitor has been defined for that node. The burn-in
period of a \JAGS\ run is therefore the interval between model
initialization and the creation of the first monitor.

When a model is initialized, it may be in {\em adaptive mode}, meaning
that the Samplers used by the model may modify their behaviour for
increased efficiency. Since this adaptation may depend on the entire
sample history, the sequence generated by an adapting sampler is no
longer a Markov chain, and is not guaranteed to converge to the target
distribution. Therefore, adaptive mode must be turned off at some
point during burn-in, and a sufficient number of iterations must take
place {\em after} the adaptive phase to ensure successful burnin.

By default, adaptive mode is turned off half way through first update
of a \JAGS\ model. All samplers have a built in test to determine
whether they have converged to their optimal sampling behaviour.  If
any sampler fails this validation test, a warning will be printed. To
ensure optimal sampling behaviour, the model should be run again from
scratch using a longer adaptation period.

The \texttt{adapt} command (see section \ref{section:adapt}) can be
used for more control over the adaptive phase.  The \texttt{adapt}
command updates the model but keeps it in adaptive mode. At the end of
each update, the convergence test is called. The message ``Adaptation
successful'' will be printed if the convergence test is successful,
otherwise the message will read ``Adaptation incomplete''.  Successive
calls to \texttt{adapt} are possible while keeping the model in
adaptive mode. The next \texttt{update} command will immediately turn
off adaptive mode.

\section{Monitoring}
\label{section:monitoring}

A {\em monitor} in \JAGS\ is an object that records sampled
values. The simplest monitor is a {\em trace monitor}, which stores
the sampled value of a node at each iteration.

\JAGS\ cannot monitor a node unless it has been defined in the model
file.  For vector- or array-valued nodes, this means that every
element must be defined. Here is an example of a simple for loop that
only defines elements $2$ to $N$ of \verb+theta+

\begin{verbatim}
for (i in 2:N) {
   theta[i] ~ dnorm(0,1);
}
\end{verbatim}

Unless \verb+theta[1]+ is defined somewhere else in the model file,
the multivariate node \verb+theta+ is undefined and therefore it
will not be possible to monitor \verb+theta+ as a whole.  In such
cases you can request each element separately , e.g. \verb+theta[2]+,
\verb+theta[3]+, {\em etc.}, or request a subset that is fully defined,
e.g. \verb+theta[2:6]+.

Monitors can be classified according to whether they pool values over
iterations and whether they pool values over parallel chains (The
standard trace monitor does neither). When monitor values are written
out to file using the CODA command, the output files created depend
on the pooling of the monitor, as shown in table \ref{table:coda}. By
default, all of these files have the prefix CODA, but this may be changed
to any other name using the ``stem'' option to the CODA command
(See \ref{coda}).

\begin{table}[h]
\begin{tabular}{ccl}
\hline
Pool       & Pool   & Output files \\
iterations & chains &              \\
\hline
no         & no     & CODAindex.txt, CODAchain1.txt, ... 
                      CODAchainN.txt \\
no         & yes    & CODAindex0.txt, CODAchain0.txt \\
yes        & no     & CODAtable1.txt, ... CODAtableN.txt \\
yes        & yes    & CODAtable0.txt \\
\hline
\end{tabular}
\caption{Output files created by the CODA command depending on whether
a monitor pools its values over chains or over iterations \label{table:coda}}
\end{table}

The standard CODA format for monitors that do not pool values over
iterations is to create an index file and one or more output files.
The index file is has three columns with, one each line,
\begin{enumerate}
\item A string giving the name of the (scalar) value being recorded
\item The first line in the output file(s)
\item The last line in the output file(s)
\end{enumerate}
The output file(s) contain two columns:
\begin{enumerate}
\item The iteration number
\item The value at that iteration
\end{enumerate}

Some monitors pool values over iterations. For example a mean monitor
may record only the sample mean of a node, without keeping the
individual values from each iteration. Such monitors are written out
to a table file with two columns:
\begin{enumerate}
\item A string giving the name of the (scalar) value being recorded
\item The value (pooled over all iterations)
\end{enumerate}

\chapter{Running \JAGS}

\JAGS\ has a command line interface. To invoke jags interactively,
simply type \texttt{jags} at the shell prompt on Unix, a Terminal window on MacOS, or the Windows
command prompt on Windows. To invoke JAGS with a script file, type
\begin{verbatim}
jags <script file>
\end{verbatim}
A typical script file has the following commands: 
\begin{verbatim}
model in "line.bug"      # Read model file
data in "line-data.R"    # Read data in from file
compile, nchains(2)      # Compile a model with two parallel chains
parameters in "line-inits1.R", chain(1) # Read initial values from file for chain 1
parameters in "line-inits2.R", chain(2) # Read initial values from file for chain 2
initialize               # Initialize the model
update 1000              # Adaptation (if necessary) and burnin for 1000 iterations
monitor alpha            # Set trace monitor for node alpha ...
monitor beta             # ... and beta
monitor sigma            # ... and sigma
update 10000             # Update model for 10000 iterations
coda *                   # All monitored values are written out to file
\end{verbatim}
More examples can be found in the file \verb+classic-bugs.tar.gz+
which is available from Sourceforge
(\url{http://sourceforge.net/projects/mcmc-jags/files}.

Output from \JAGS\ is printed to the standard output, even when a
script file is being used.  

\section{Scripting commands}
\label{section:scripting}

\JAGS\ has a simple set of scripting commands with a syntax loosely
based on \textsf{Stata}. Commands are shown below preceded by a dot
(.). This is the \JAGS\ prompt. Do not type the dot in when you are
entering the commands.

C-style block comments taking the form /* ... */ can be
embedded anywhere in the script file.  Additionally, you may use
\R-style single-line comments starting with \#.

If a scripting command takes a file name, then the name may be
optionally enclosed in quotes. Quotes are required when the file name
contains space, or any character which is not alphanumeric, or one of
the following: \verb+_+, \verb+-+, \verb+.+, \verb+/+, \verb+\+.

In the descriptions below, angular brackets \verb+<>+, and the text
inside them, represents a parameter that should be replaced with the
correct value by you.  Anything inside square brackets \verb+[]+ is
optional. Do not type the square brackets if you wish to use an
option.

\subsection{MODEL IN}

\begin{verbatim}
. model in <file>
\end{verbatim}
Checks the syntactic correctness of the model description in
\texttt{file} and reads it into memory. The next compilation
statement will compile this model. 

See also: MODEL CLEAR (\ref{model:clear})

\subsection{DATA IN}
\label{data:in}

\begin{verbatim}
. data in <file>
\end{verbatim}
JAGS keeps an internal data table containing the values of observed
nodes inside each node array.  The DATA IN statement reads data from a
file into this data table.

Several data statements may be used to read in data from more than one
file. If two data files contain data for the same node array, the second
set of values will overwrite the first, and a warning will be printed.

See also: DATA TO (\ref{data:to}).

\subsection{COMPILE}

\begin{verbatim}
. compile [, nchains(<n>)]
\end{verbatim}
Compiles the model using the information provided in the preceding
model and data statements. By default, a single Markov chain is
created for the model, but if the \texttt{nchains} option is given,
then \texttt{n} chains are created 

Following the compilation of the model, further DATA IN statements are
legal, but have no effect.  A new model statement, on the other hand,
will replace the current model.

\subsection{PARAMETERS IN}
\label{parameters:in}

\begin{verbatim}
. parameters in <file> [, chain(<n>)]
\end{verbatim}
Reads the values in \texttt{file} and writes them to the corresponding
parameters in chain \texttt{n}. The file has the same format as the
one in the DATA IN statement.  The \texttt{chain} option may be
omitted, in which case the parameter values in all chains are set to
the same value.

The PARAMETERS IN statement must be used after the COMPILE statement
and before the INITIALIZE statement.  You may only supply the values of
unobserved stochastic nodes in the parameters file, not logical or
constant nodes.

See also: PARAMETERS TO (\ref{parameters:to})

\subsection{INITIALIZE}

\begin{verbatim}
. initialize
\end{verbatim}
Initializes the model using the previously supplied data and parameter
values supplied for each chain.

\subsection{UPDATE}

\begin{verbatim}
. update <n> [,by(<m>)]
\end{verbatim}
Updates the model by \texttt{n} iterations. 

If JAGS is being run interactively, a progress bar is printed on the
standard output consisting of 50 asterisks. If the \texttt{by} option
is supplied, a new asterisk is printed every \texttt{m} iterations. If
this entails more than 50 asterisks, the progress bar will be wrapped
over several lines.  If \texttt{m} is zero, the printing of the
progress bar is suppressed.

If JAGS is being run in batch mode, then the progress bar is
suppressed by default, but you may activate it by supplying the
\texttt{by} option with a non-zero value of \texttt{m}.

If the model has an adaptive sampling phase, the first UPDATE
statement turns off adaptive mode for all samplers in the model after
\texttt{n/2} iterations. A warning is printed if adaptation is
incomplete. Incomplete adaptation means that the mixing of the Markov
chain is not optimal. It is still possible to continue with a model
that has not completely adapted, but it may be preferable to run the
model again with a longer adaptation phase, starting from the MODEL IN
statement. Alternatively, you may use an ADAPT statement (see below)
immediately after initialization.

\subsection{ADAPT}
\label{section:adapt}

\begin{verbatim}
. adapt <n> [,by(<m>)]
\end{verbatim}
Updates the model by \texttt{n} iterations keeping the model in
adaptive mode and prints a message to indicate whether adaptation is
successful.  Successive calls to ADAPT may be made until the
adaptation is successful. The next call to UPDATE then turns off
adaptive mode immediately.

Use this instead of the first UPDATE statement if you want explicit
control over the length of the adaptive sampling phase.

Like the UPDATE statement, the ADAPT statement prints a progress
bar, but with plus signs instead of asterisks.

\subsection{MONITOR}
\label{section:monitor}

In \JAGS, a monitor is an object that calculates summary statistics
from a model.  The most commonly used monitor simply records the value
of a single node at each iteration.  This is called a ``trace''
monitor. 
\begin{verbatim}
. monitor <varname> [, thin(n)] [type(<montype>)]
\end{verbatim}
The \texttt{thin} option sets the thinning interval of the monitor so
that it will only record every nth value. The \texttt{thin} option 
selects the type of monitor to create. The default type is \texttt{trace}.

More complex monitors can be defined that do additional calculations.
For example, the \verb+dic+ module defines a ``deviance'' monitor that
records the deviance of the model at each iteration, and a ``pD''
monitor that calculates an estimate of the effective number of
parameters on the model \cite{spiegelhalter:etal:2002}.  

\begin{verbatim}
. monitor clear <varname> [type(<montype>)]
\end{verbatim}
Clears the monitor of the given type associated with variable
\texttt{<varname>}. 

\subsection{CODA}
\label{coda}

\begin{verbatim}
. coda <varname> [, stem(<filename>)]
\end{verbatim}
If the named node has a trace monitor, this dumps the monitored values
of to files \texttt{CODAindex.txt}, \texttt{CODAindex1.out},
\texttt{CODAindex2.txt}, \ldots in a form that can be read by the
\CODA\ package of \R.  The stem option may be used to modify the
prefix from ``CODA'' to another string.  The wild-card character ``*''
may be used to dump all monitored nodes

\subsection{EXIT}

\begin{verbatim}
. exit
\end{verbatim}
Exits \JAGS. \JAGS\ will also exit when it reads an end-of-file character.

\subsection{DATA TO}
\label{data:to}
\begin{verbatim}
. data to <filename>
\end{verbatim}
Writes the data ({\em i.e.} the values of the observed nodes) to a
file in the R \texttt{dump} format. The same file can be used in a
DATA IN statement for a subsequent model.

See also: DATA IN (\ref{data:in})

\subsection{PARAMETERS TO}
\label{parameters:to}
\begin{verbatim}
. parameters to <file> [, chain(<n>)]
\end{verbatim}
Writes the current parameter values ({\em i.e.} the values of the
unobserved stochastic nodes) in chain \texttt{<n>} to a file in R dump
format. The name and current state of the RNG for chain \texttt{<n>}
is also dumped to the file.  The same file can be used as input in a
PARAMETERS IN statement in a subsequent run.

See also: PARAMETERS IN (\ref{parameters:in})

\subsection{SAMPLERS TO}
\label{samplers:to}
\begin{verbatim}
. samplers to <file>
\end{verbatim}
Writes out a summary of the samplers to the given file.  The output appears
in three tab-separated columns, with one row for each sampled node
\begin{itemize}
\item The index number of the sampler (starting with 1). The index number 
gives the order in which Samplers are updated at each iteration.
\item The name of the sampler, matching the index number
\item The name of the sampled node. 
\end{itemize}
If a Sampler updates multiple nodes then it is represented by multiple rows
with the same index number.

Note that the list includes only those nodes that are updated by a
Sampler.  Stochastic nodes that are updated by forward sampling from
the prior are not listed.

\subsection{LOAD}
\label{load}
\begin{verbatim}
. load <module>
\end{verbatim}
Loads a module into \JAGS\ (see chapter \ref{section:modules}). Loading
a module does not affect any previously initialized models, but will
affect the future behaviour of the compiler and model initialization.

\subsection{UNLOAD}
\label{unload}
\begin{verbatim}
. unload <module>
\end{verbatim}
Unloads a module. Currently initialized models are unaffected, but
the functions, distribution, and factory objects in the model will not
be accessible to future models.

\subsection{LIST MODULES}
\label{list:modules}
\begin{verbatim}
. list modules
\end{verbatim}
Prints a list of the currently loaded modules.

\subsection{LIST FACTORIES}
\label{list:factories}
\begin{verbatim}
. list factories, type(<factype>)
\end{verbatim}
List the currently loaded factory objects and whether or not they are
active.  The \verb+type+ option must be given, and the three possible values
of \verb+<factype>+ are \verb+sampler+, \verb+monitor+, and \verb+rng+.

\subsection{SET FACTORY}
\label{set:factory}
\begin{verbatim}
. set factory "<facname>" <status>, type(<factype>)
\end{verbatim}
Sets the status of a factor object. The possible values of \verb+<status>+
are \verb+on+ and \verb+off+. Possible factory names are given from the
LIST MODULES command.

\subsection{MODEL CLEAR}
\label{model:clear}
\begin{verbatim}
. model clear
\end{verbatim}
Clears the current model.  The data table (see section \ref{data:in})
remains intact

\subsection{Print Working Directory (PWD)}
\begin{verbatim}
. pwd
\end{verbatim}
Prints the name of the current working directory. This is where \JAGS\
will look for files when the file name is given without a full path, 
{\em e.g.} \verb+"mymodel.bug"+.

\subsection{Change Directory (CD)}
\begin{verbatim}
. cd <dirname>
\end{verbatim}
Changes the working directory to \texttt{<dirname>}

\subsection{Directory list (DIR)}
\begin{verbatim}
. dir
\end{verbatim}
Lists the files in the current working directory.

\subsection{RUN}
\begin{verbatim}
. run <cmdfile>
\end{verbatim}
Opens the file \texttt{<cmdfile>} and reads further scripting commands
until the end of the file.  Note that if the file contains an EXIT
statement, then the \JAGS\ session will terminate. 


\section{Errors}

There are two kinds of errors in \JAGS: runtime errors, which are due to
mistakes in the model specification, and logic errors which are internal
errors in the JAGS program. 

Logic errors are generally created in the lower-level parts of the \JAGS\
library, where it is not possible to give an informative error message.
The upper layers of the \JAGS\ program are supposed to catch such errors
before they occur, and return a useful error message that will help you
diagnose the problem.  Inevitably, some errors slip through. Hence,
if you get a logic error, there is probably an error in your input to
\JAGS, although it may not be obvious what it is. Please send a bug
report (see ``Feedback'' below) whenever you get a logic error.

Error messages may also be generated when parsing files (model files,
data files, command files).  The error messages generated in this case
are created automatically by the program \texttt{bison}. They
generally take the form ``syntax error, unexpected FOO, expecting BAR''
and are not always abundantly clear.

If a model compiles and initializes correctly, but an error occurs
during updating, then the current state of the model will be dumped
to a file named \verb+jags.dumpN.R+ where $N$ is the chain number.
You should then load the dumped data into R to inspect the state of
each chain when the error occurred.

\chapter{Modules}
\label{section:modules}

The \JAGS\  library is distributed along with certain dynamically
loadable modules that extend its functionality. A module can define
new objects of the following classes:
\begin{enumerate}
\item {\bf functions} and {\bf distributions}, the basic building
blocks of the BUGS language.
\item {\bf samplers}, the objects which update the parameters of the
model at each iteration, and {\bf sampler factories}, the objects that 
create new samplers for specific model structures.  If the module
defines a new distribution, then it will typically also define a new
sampler for that distribution.
\item {\bf monitors}, the objects that record sampled values for
later analysis, and {\bf monitor factories} that create them. 
\item {\bf random number generators}, the objects that drive the
MCMC algorithm and {\bf RNG factories} that create them.
\end{enumerate}

The \verb+base+ module and the \verb+bugs+ module are loaded automatically
at start time.  Others may be loaded by the user.

\section{The base module}

The base module supply the base functionality for the \JAGS\ library
to function correctly. It is loaded first by default.  

\subsection{Base Samplers}

The \verb+base+ module defines samplers that use highly generic update
methods.  These sampling methods only require basic information about
the stochastic nodes they sample.  Conversely, they may not be fully
efficient.

Three samplers are currently defined:
\begin{enumerate}
\item The Finite sampler can sample a discrete-valued node with
fixed support of less than 20 possible values. The node must not
be bounded using the \verb+T(,)+ construct
\item The Real Slice Sampler can sample any scalar real-valued 
stochastic node.
\item The Discrete Slice Sampler can sample any scalar
discrete-valued stochastic node.
\end{enumerate}

\subsection{Base RNGs}

The \verb+base+ module defines four RNGs, taken directly from \R,
with the following names:
\begin{enumerate}
\item \verb+"base::Wichmann-Hill"+
\item \verb+"base::Marsaglia-Multicarry"+
\item \verb+"base::Super-Duper"+
\item \verb+"base::Mersenne-Twister"+
\end{enumerate}

A single RNG factory object is also defined by the \verb+base+
module which will supply these RNGs for chains 1 to 4 respectively, if
``RNG.name'' is not specified in the initial values file.  All chains
generated by the base RNG factory are initialized using the current
time stamp.

If you have more than four parallel chains, then the base module will
recycle the same for RNGs, but using different seeds. If you want many
parallel chains then you may wish to load the \verb+lecuyer+ module.

\subsection{Base Monitors}

The \verb+base+ module defines the TraceMonitor class (type
``trace''). This is the monitor class that simply records the current
value of the node at each iteration.

\section{The bugs module}

The \verb+bugs+ module defines some of the functions and distributions
from \OpenBUGS. These are described in more detail in sections
\ref{section:functions} and \ref{section:distributions}.  The
\verb+bugs+ module also defines conjugate samplers for efficient Gibbs
sampling.

\section{The mix module}

The \verb+mix+ module defines a novel distribution
\verb+dnormmix(mu,tau,pi)+ representing a finite mixture of normal
distributions. In the parameterization of the \verb+dnormmix+
distribution, $\mu$, $\tau$, and $\pi$ are vectors of the same length,
and the density of \verb+y ~ dnormmix(mu, tau, pi)+ is
\[
f(y | \mu, \tau, \pi) = \sum_i \pi_i \tau_i^{\frac{1}{2}} \phi( \tau^{\frac{1}{2}}_i (y - \mu_i))
\]
where $\phi()$ is the probability density function of a standard
normal distribution.

The \verb+mix+ module also defines a sampler that is designed to act
on finite normal mixtures. It uses tempered transitions to jump
between distant modes of the multi-modal posterior distribution
generated by such models \cite{Neal94,Celeux99}. The tempered
transition method is computationally very expensive. If you want to
use the \verb+dnormmix+ distribution but do not care about label
switching, then you can disable the tempered transition sampler with
\begin{verbatim}
set factory "mix::TemperedMix" off, type(sampler)
\end{verbatim}

\section{The dic module}

The \verb+dic+ module defines new monitor classes for Bayesian model
criticism using deviance-based measures. 

\subsection{The deviance monitor}

The deviance monitor records the deviance of the model ({\em i.e.} the
sum of the deviances of all the observed stochastic nodes) at each
iteration. The command
\begin{verbatim}
monitor deviance
\end{verbatim}
will create a deviance monitor {\em unless} you have defined a node
called ``deviance'' in your model. In this case, you will get a trace
monitor for your deviance node.

\subsection{The \texttt{pD} monitor}

The \verb+pD+ monitor is used to estimate the effective number of
parameters ($p_D$) of the model \cite{spiegelhalter:etal:2002}. It
requires at least two parallel chains in the model, but calculates
a single estimate of $p_D$ across all chains \cite{plummer:2002}.
A pD monitor can be created using the command:
\begin{verbatim}
monitor pD
\end{verbatim}
Like the deviance monitor, however, if you have defined a node called
``pD'' in your model then this will take precedence, and you will get
a trace monitor for your \verb+pD+ node.

Since the $p_D$ monitor pools its value across all chains, its values
will be written out to the index file ``CODAindex0.txt'' and
output file ``CODAoutput0.txt'' when you use the CODA command.

The effective number of parameters is the sum of separate contributions
from all observed stochastic nodes: $p_D = \sum_i p_{D_i}$. There is
also a monitor that stores the sample mean of $p_{D_i}$. These statistics
may be used as influence diagnostics \cite{spiegelhalter:etal:2002}.
The mean monitor for $p_{D_i}$ is created with:
\begin{verbatim}
monitor pD, type(mean)
\end{verbatim}
Its values can be written out to a file ``PDtable0.txt'' with
\begin{verbatim}
coda pD, type(mean) stem(PD)
\end{verbatim}

\subsection{The \texttt{popt} monitor}

The \verb+popt+ monitor works exactly like the mean monitor for $p_D$,
but records contributions to the optimism of the expected deviance
($p_{opt_i}$). The total optimism $p_{opt} = \sum_i p_{opt_i}$  can be
added to the mean deviance to give the penalized expected deviance
\cite{plummer:2008}.

The mean monitor for $p_{opt_i}$ is created with
\begin{verbatim}
monitor popt, type(mean)
\end{verbatim}
Its values can be written out to a file ``POPTtable0.txt'' with
\begin{verbatim}
coda popt, type(mean) step(POPT)
\end{verbatim}
Under asymptotically favourable conditions in which $p_{D_i} \ll 1
\forall i$,
\[
p_{opt} \approx 2 p_D
\]
For generalized linear models, a better approximation is
\[
p_{opt} \approx \sum_{i=1}^n \frac {p_{D_i}}{1 - p_{D_i}}
\]

The \verb+popt+ monitor uses importance weights to estimate
$p_{opt}$. The resulting estimates may be numerically unstable when
$p_{D_i}$ is not small.  This typically occurs in random-effects
models, so it is recommended to use caution with the \verb+popt+
until I can find a better way of estimating $p_{opt_i}$.

\section{The msm module}

The \verb+msm+ module defines the matrix exponential function
\verb+mexp+ and the multi-state distribution \verb+dmstate+ which
describes the transitions between observed states in continuous-time
multi-state Markov transition models. 

\section{The glm module}

The \verb+glm+ module implements samplers for efficient updating of
generalized linear mixed models.  The fundamental idea is to do block
updating of the parameters in the linear predictor.  The \verb+glm+
module is built on top of the \textsf{Csparse} and \textsf{CHOLMOD}
sparse matrix libraries
\cite{Davis2006, Davis1999} which allows updating of both fixed and random
effects in the same block. Currently, the methods only work on
parameters that have a normal prior distribution.

Some of the samplers are based in the idea of introducing latent
normal variables that reduce the GLM to a linear model. This idea was
introduced by Albert and Chib \cite{AlbertChib93} for probit
regression with a binary outcome, and was later refined and extended
to logistic regression with binary outcomes by Holmes and Held
\cite{HolmesHeld06}. Another approach, auxiliary mixture sampling,
was developed by Fr{\"u}hwirth-Schnatter {\em et al}
\cite{Fruhwirth-Schnatter09} and is used for more general Poisson
regression and logistic regression models with binomial outcomes.
Gamerman \cite{Gamerman97} proposed a stochastic version of the iteratively
weighted least squares algorithm for GLMs, which is also implemented
in the \verb+glm+ module. However the IWLS sampler tends to break down
when there are many random effects in the model. It uses
Metropolis-Hastings updates, and the acceptance probability may be
very small under these circumstances.

Block updating in GLMMs frees the user from the need to center
predictor variables, like this:
\begin{verbatim}
y[i] ~ dnorm(mu[i], tau)
mu[i] <- alpha + beta * (x[i] - mean(x))
\end{verbatim}
The second line can simply be written
\begin{verbatim}
mu[i] <- alpha + beta * x[i]
\end{verbatim}
without affecting the mixing of the Markov chain.  

\chapter{Functions}
\label{section:functions}

Functions allow deterministic nodes to be defined using the \verb+<-+
(``gets'') operator.  Most of the functions in \JAGS\ are scalar
functions taking scalar arguments. However, \JAGS\ also allows
arbitrary vector- and array-valued functions, such as the matrix
multiplication operator \verb+%*%+ and the transpose function
\verb+t()+ defined in the \verb+bugs+ module, and the matrix
exponential function \verb+mexp()+ defined in the \verb+msm+
module. \JAGS\ also uses an enriched dialect of the BUGS language with
a number of operators that are used in the S language.

Scalar functions taking scalar arguments are automatically vectorized.
They can also be called when the arguments are arrays with conforming
dimensions, or scalars. So, for example, the scalar $c$ can be added to
the matrix $A$ using
\begin{verbatim}
B <- A + c
\end{verbatim}
instead of the more verbose form
\begin{verbatim}
D <- dim(A)
for (i in 1:D[1])
   for (j in 1:D[2]) {
      B[i,j] <- A[i,j] + c
   }
}
\end{verbatim}
Inverse link functions are an exception to this rule ({\em i.e.} exp,
icloglog, ilogit, phi) and cannot be vectorized. 

\section{Base functions}
\label{section:functions:base}

The functions defined by the \verb+base+ module all appear as infix or
prefix operators. The syntax of these operators is built into the
\JAGS\ parser. They are therefore considered part of the modelling
language.  Table \ref{table:base:functions} lists them in reverse
order of precedence.

\begin{table}[h]
\begin{center}
\begin{tabular}{lll}
\hline
Type & Usage & Description\\ 
\hline
Logical           & \verb+x || y+ & Or \\
operators         & \verb+x && y+ & And \\
                  & \verb+!x+     & Not \\
\hline
Comparison  & \verb+x > y+ & Greater than\\
operators   & \verb+x >= y+ & Greater than or equal to  \\
            & \verb+x < y+ & Less than \\
            & \verb+x <= y+ & Less than or equal to \\
            & \verb+x == y+ & Equal \\
            & \verb+x != y+ & Not equal \\
\hline
Arithmetic  & \verb-x + y- & Addition \\
operators   & \verb+x - y+ & Subtraction\\
            & \verb+x * y+ & Multiplication \\
            & \verb+x / y+ & Division \\
            & \verb+x %special% y+ &User-defined operators\\
            & \verb+-x+ & Unary minus\\
\hline
Power function & \verb+x^y+ & \\
\hline
\end{tabular}
\caption{Base functions listed in reverse order of precedence 
  \label{table:base:functions}}
\end{center}
\end{table}

Logical operators convert numerical arguments to logical values: zero
arguments are converted to FALSE and non-zero arguments to
TRUE. Logical and comparison operators return the value 1 if the
result is TRUE and 0 if the result is FALSE.  Comparison operators are
non-associative: the expression \verb+x < y < z+, for example, is
syntactically incorrect.

The \verb+%special%+ function is an exception in table
\ref{table:base:functions}. It is not a function defined by the
\verb+base+ module, but is a place-holder for any function
with a name starting and ending with the character ``\verb+%+'' Such
functions are automatically recognized as infix operators by the
\JAGS\ model parser, with precedence defined by table
\ref{table:base:functions}.

\section{Functions in the bugs module}
\label{section:functions:bugs}

\subsection{Scalar functions}

Table \ref{table:bugs:scalar} lists the scalar-valued functions in the
\texttt{bugs} module that also have scalar arguments.  These functions
are automatically vectorized when they are given vector, matrix, or
array arguments with conforming dimensions.

Table \ref{table:bugs:link} lists the link functions in the
\texttt{bugs} module.  These are smooth scalar-valued functions that
may be specified using an S-style replacement function notation. So,
for example, the log link
\begin{verbatim}
log(y) <- x
\end{verbatim}
is equivalent to the more direct use of its inverse, the exponential
function:
\begin{verbatim}
y <- exp(x)
\end{verbatim}
This usage comes from the use of link functions in generalized linear
models.

Table \ref{table:bugs:dpq} shows functions to calculate the
probability density, probability function, and quantiles of some of
the distributions provided by the \texttt{bugs} module. These
functions are parameterized in the same way as the corresponding
distribution.  For example, if $x$ has a normal distribution with mean
$\mu$ and precision $\tau$
\begin{verbatim}
x ~ dnorm(mu, tau)
\end{verbatim}
Then the usage of the corresponding density, probability, and quantile
functions is: 
\begin{verbatim}
density.x  <- dnorm(x, mu, tau)     # Density of normal distribution at x
prob.x     <- pnorm(x, mu, tau)     # P(X <= x)
quantile90.x <- qnorm(0.9, mu, tau) # 90th percentile
\end{verbatim}
For details of the parameterization of the other distributions, see
tables \ref{table:bugs:distributions:real} and
\ref{table:bugs:distributions:discrete}.

\begin{table}
\begin{center}
\begin{tabular}{llll}
\hline
Usage  & Description & Value & Restrictions on arguments \\ 
\hline
\verb+abs(x)+       & Absolute value        & Real & \\
\verb+arccos(x)+    & Arc-cosine            & Real & $-1 < x < 1$\\
\verb+arccosh(x)+   & Hyperbolic arc-cosine & Real & $1 < x$ \\
\verb+arcsin(x)+    & Arc-sine              & Real & $-1 < x < 1$\\
\verb+arcsinh(x)+   & Hyperbolic arc-sine   & Real &\\
\verb+arctan(x)+    & Arc-tangent           & Real &\\
\verb+arctanh(x)+   & Hyperbolic arc-tangent & Real & $-1 < x < 1$\\
\verb+cos(x)+       & Cosine              & Real & \\
\verb+cosh(x)+      & Hyperbolic Cosine   & Real & \\
\verb+cloglog(x)+    & Complementary log log & Real & $0 < x < 1$ \\
\verb+equals(x,y)+   & Test for equality   & Logical & \\
\verb+exp(x)+       & Exponential         & Real & \\
\verb+icloglog(x)+  & Inverse complementary & Real & \\
                    & log log function    & \\
\verb+ifelse(x,a,b)+ & If $x$ then $a$ else $b$ & Real & \\
\verb+ilogit(x)+    & Inverse logit       & Real & \\
\verb+log(x)+       & Log function        & Real & $x > 0$ \\
\verb+logfact(x)+   & Log factorial       & Real & $x > -1$ \\
\verb+loggam(x)+    & Log gamma           & Real & $x > 0$ \\
\verb+logit(x)+     & Logit               & Real & $0 < x < 1$ \\
\verb+phi(x)+       & Standard normal cdf & Real & \\
\verb+pow(x,z)+     & Power function      & Real & If $x < 0$ then $z$ is integer \\ 
\verb+probit(x)+    & Probit              & Real & $0 < x < 1$ \\
\verb+round(x)+     & Round to integer    & Integer & \\
                    & away from zero      &      & \\
\verb+sin(x)+       & Sine                & Real & \\
\verb+sinh(x)+      & Hyperbolic Sine     & Real & \\
\verb+sqrt(x)+      & Square-root         & Real & $x >= 0$ \\
\verb+step(x)+      & Test for $x \geq 0$ & Logical & \\
\verb+tan(x)+       & Tangent             & Real & \\
\verb+tanh(x)+      & Hyperbolic Tangent  & Real & \\
\verb+trunc(x)+     & Round to integer    & Integer & \\
                    & towards zero        & \\
\hline
\end{tabular}
\caption{Scalar functions in the \texttt{bugs} module \label{table:bugs:scalar}}
\end{center}
\end{table}

\begin{table}
\begin{center}
\begin{tabular}{llll}
\hline
Distribution & Density & Distribution & Quantile \\
\hline
Bernoulli          & dbern     & pbern     & qbern \\
Beta               & dbeta     & pbeta     & qbeta \\
Binomial           & dbin      & pbin      & qbin \\
Chi-square         & dchisqr   & pchisqr   & qchisqr \\
Double exponential & ddexp     & pdexp     & qdexp \\
Exponential        & dexp      & pexp      & qexp \\
F                  & df        & pf        & qf \\
Gamma              & dgamma    & pgamma    & qgamma \\
Generalized gamma  & dgen.gamma & pgen.gamma & qgen.gamma \\
Noncentral hypergeometric     & dhyper    & phyper    & qhyper \\
Logistic           & dlogis    & plogis    & qlogis \\
Log-normal         & dlnorm    & plnorm    & qlnorm \\
Negative binomial  & dnegbin   & pnegbin   & qnegbin \\
Noncentral Chi-square & dnchisqr   & pnchisqr   & qnchisqr \\
Normal             & dnorm     & pnorm     & qnorm \\
Pareto             & dpar      & ppar      & qpar \\
Poisson            & dpois     & ppois     & qpois \\
Student t          & dt        & pt        & qt \\
Weibull            & dweib     & pweib     & qweib \\
\hline
\end{tabular}
\caption{Functions to calculate the probability density, probability
  function, and quantiles of some of the distributions provided by the
  \texttt{bugs} module. \label{table:bugs:dpq}}
\end{center}
\end{table}

\begin{table}
\begin{center}
\begin{tabular}{llll}
\hline
Link function         & Description & Range & Inverse \\
\hline
\verb+cloglog(y) <- x+ & Complementary log log & $0 < y < 1$ & \verb+y <- icloglog(x)+ \\
\verb+log(y) <- x+    & Log           & $0 < y$ &  \verb+y <- exp(x)+ \\
\verb+logit(y) <- x+  & Logit         & $0 < y < 1$ &  \verb+y <- ilogit(x)+ \\
\verb+probit(y) <- x+ & Probit        & $0 < y < 1$ &  \verb+y <- phi(x)+\\
\hline
\end{tabular}
\caption{Link functions in the \texttt{bugs} module \label{table:bugs:link}}
\end{center}
\end{table}

\subsection{Scalar-valued functions with vector arguments}

Table \ref{table:bugs:scalar2} lists the scalar-valued functions in the
\texttt{bugs} module that take general arguments. Unless otherwise
stated in table \ref{table:bugs:scalar2}, the arguments to these functions
may be scalar, vector, or higher-dimensional arrays.

The \verb+max()+ and \verb+min()+ functions work like the
corresponding \R\ functions. They take a variable number of arguments
and return the maximum/minimum element over all supplied
arguments. This usage is  compatible with \OpenBUGS, although more general.

\begin{table}
\begin{tabular}{lll}
\hline
Function & Description & Restrictions \\
\hline
\verb+inprod(x1,x2)+ & Inner product & Dimensions of $x1$, $x2$ conform \\
\verb+interp.lin(e,v1,v2)+ & Linear Interpolation & $e$ scalar, \\
                          &                     & $v1,v2$ conforming vectors \\
\verb+logdet(m)+ & Log determinant & $m$ is a symmetric positive definite matrix \\
\verb+max(x1,x2,...)+ & Maximum element among all arguments & \\
\verb+mean(x)+  & Mean of elements of $x$ & \\
\verb+min(x1,x2,...)+ & Minimum element among all arguments & \\
\verb+prod(x)+  & Product of elements of $x$ & \\
\verb+sum(x)+   & Sum of elements of $x$& \\
\verb+sd(x)+    & Standard deviation of elements of $x$ & \\
\hline
\end{tabular}
\caption{Scalar-valued functions with general
  arguments in the \texttt{bugs} module \label{table:bugs:scalar2}}
\end{table}

\subsection{Vector- and array-valued functions}

Table \ref{table:bugs:vector} lists vector- or matrix-valued functions
in the \texttt{bugs} module.

The \texttt{sort} and \texttt{rank} functions behaves like their R
namesakes: \texttt{sort} accepts a vector and returns the same values
sorted in ascending order; \texttt{rank} returns a vector of ranks.
This is distinct from \OpenBUGS, which has two scalar-valued functions
\verb+rank+ and \verb+ranked+.

\begin{table}
\begin{center}
\begin{tabular}{lll}
\hline
Usage & Description & Restrictions \\
\hline
\verb+inverse(a)+ & Matrix inverse & $a$ is a symmetric positive definite matrix  \\
%\verb+mexp(a)+ & Matrix exponential & $a$ is a square matrix \\
\verb+rank(v)+ & Ranks of elements of $v$ & $v$ is a vector   \\
\verb+sort(v)+ & Elements of $v$ in order & $v$ is a vector  \\
\verb+t(a)+    & Transpose                & $a$ is a matrix \\
\verb+a %*% b+  & Matrix multiplication & $a,b$ conforming vector or matrices\\

\hline
\end{tabular}
\caption{Vector- or matrix-valued functions in the \texttt{bugs}
  module \label{table:bugs:vector}}
\end{center}
\end{table}

\section{Function aliases}

A function may optionally have an alias, which can be used in the
model definition in place of the canonical name. Aliases are used to
to avoid confusion with other software in which functions may have
different names. Table \ref{table:bugs:functions:alias} shows the
functions in the \texttt{bugs} module with an alias.

\begin{table}
\begin{center}
\begin{tabular}{llll}
\hline
Function               & Canonical & Alias & Compatible  \\
                       & name      &       & with         \\
\hline
Arc-cosine             & arccos    & acos  & R \\
Hyperbolic arc-cosine  & arccosh   & acosh & R \\
Arc-sine               & arcsin    & asin  & R \\
Hyperbolic arc-sine    & arcsinh   & asinh & R \\
Arc-tangent            & arctan    & atan  & R \\
\hline
\end{tabular}
\caption{Functions with aliases in \texttt{bugs} module
  \label{table:bugs:functions:alias}}
\end{center}
\end{table}

\chapter{Distributions}
\label{section:distributions}

Distributions are used to define stochastic nodes using the \verb+~+
operator. The distributions defined in the bugs module are listed in
table \ref{table:bugs:distributions:real} (real-valued distributions),
\ref{table:bugs:distributions:discrete} (discrete-valued
distributions), and \ref{table:bugs:distributions:multi}
(multivariate distributions).

Some distributions have restrictions on the valid parameter values,
and these are indicated in the tables. If a Distribution is
given invalid parameter values when evaluating the log-likelihood, it
returns $-\infty$. When a model is initialized, all stochastic nodes
are checked to ensure that the initial parameter values are valid for
their distribution.

\begin{table}
  \begin{center}
    \begin{tabular}{llcll}
      \hline
      Name & Usage & Density & Lower & Upper \\
      \hline
      Beta & \verb+dbeta(a,b)+ & 
      \multirow{2}{*}{
        $\frac{\textstyle x^{a-1}(1-x)^{b-1}}{\textstyle \beta(a,b)}$
      } & $0$ & $1$ \\
      & $a > 0$, $b > 0$ \\
      Chi-square & \verb+dchisqr(k)+ & 
      \multirow{2}{*}{
        $\frac{\textstyle x^{\frac{k}{2} - 1} \exp(-x/2)}
        {\textstyle 2^{\frac{k}{2}} \Gamma({\scriptstyle \frac{k}{2}})}$
      } & 0 & \\
      & $k > 0$ \\
      Double  & \verb+ddexp(mu,tau)+ & 
      \multirow{2}{*}{$\tau \exp(-\tau | x-\mu |)/2$} & & \\
      exponential & $\tau > 0$ \\
      Exponential & \verb+dexp(lambda)+ & 
      \multirow{2}{*}{$\lambda \exp(-\lambda x)$} & 0 & \\ 
      & $\lambda > 0$ \\
      F   & \verb+df(n,m)+ & 
      \multirow{2}{*}{
        $\textstyle \frac{\Gamma(\frac{n + m}{2})}
                         {\Gamma(\frac{n}{2}) \Gamma(\frac{m}{2})}
        \left(\frac{n}{m} \right)^{\frac{n}{2}} x^{\frac{n}{2} - 1} 
        \left\{1 + \frac{nx}{m} \right\}^{-\frac{(n + m)}{2}}$} & 0 & \\
      & $n > 0$, $m > 0$ \\
      Gamma       & \verb+dgamma(r, lambda)+ & 
      \multirow{2}{*}{
        $\frac{\textstyle \lambda^r x^{r - 1} \exp(-\lambda x)}
        {\textstyle \Gamma(r)}$} & 0 & \\
      & $\lambda > 0$, $r > 0$ \\
      Generalized & \verb+dgen.gamma(r,lambda,b)+ &  
      \multirow{2}{*}{
        $\frac
        {\textstyle b \lambda^{b r} x^{b r - 1}  \exp\{-(\lambda x)^{b}\}}
        {\textstyle \Gamma(r)}$
      } & $0$ & \\
      gamma       & $\lambda >0$, $b > 0$, $r > 0$ \\
      Logistic    & \verb+dlogis(mu, tau)+ &
      \multirow{2}{*}{
        $\frac{\textstyle \tau \exp\{(x - \mu) \tau\}}
        {\textstyle  \left[1 + \exp\{(x - \mu) \tau\}\right]^2}$
      } &  & \\
      ~ & $\tau > 0$ \\
      Log-normal  & \verb+dlnorm(mu,tau)+ & 
      \multirow{2}{*}{
        $\left(\frac{\tau}{2\pi}\right)^{\frac{1}{2}} x^{-1} \exp \left\{-\tau (\log(x) - \mu)^2 / 2 \right\}$} & 0 \\
      ~ & $\tau > 0$ \\
      Noncentral & \verb+dnchisqr(k, delta)+ & 
      \multirow{2}{*}{
        $\sum_{r=0}^{\infty} 
        \frac{ \exp(-\frac{\delta}{2}) (\frac{\delta}{2})^r}{\textstyle r!} \,
        \frac{ x^{(k/2 + r - 1)} \exp(-\frac{x}{2})}
             { 2^{(k/2 + r)} \Gamma({ \frac{k}{2} + r})}
        $
      } & 0 & \\
      Chi-squre & $k > 0, \delta \geq 0$ \\
      Normal   & \verb+dnorm(mu,tau)+ & 
      \multirow{2}{*}{
        $\left(\frac{\tau}{2\pi}\right)^{\frac{1}{2}} \exp\{- \tau (x - \mu)^2 / 2\}$} & & \\
      ~ & $\tau > 0$ \\
      Pareto      & \verb+dpar(alpha, c)+ & 
      \multirow{2}{*}{
        $\alpha c^{\alpha} x^{-(\alpha + 1)}$
      } & $c$ & \\
      ~ & $\alpha > 0$, $c > 0$ \\
      Student t   & \verb+dt(mu,tau,k)+ & 
      \multirow{2}{*}{
        $\textstyle \frac{\Gamma(\frac{k+1}{2})}{\Gamma(\frac{k}{2})} 
        \left(\frac{\tau}{k\pi} \right)^{\frac{1}{2}} 
        \left\{1 + \frac{\tau (x - \mu)^2}{k} \right\}^{-\frac{(k+1)}{2}}$} & & \\
      ~ & $\tau > 0$, $k > 0$ \\
      Uniform     & \verb+dunif(a,b)+ & 
      \multirow{2}{*}{$\frac{\textstyle 1}{\textstyle b - a}$} & $a$ & $b$ \\
      ~ & $a < b$ \\ 
      Weibull     & \verb+dweib(v, lambda)+ & 
      \multirow{2}{*}{$v  \lambda  x^{v - 1} \exp (- \lambda x^v)$} & 0 & \\
      ~ & $v > 0$, $\lambda > 0$ \\
      \hline
    \end{tabular}
    \caption{Univariate real-valued distributions in the \texttt{bugs} module
      \label{table:bugs:distributions:real}}
  \end{center}
\end{table}

\begin{table}
  \begin{center}
    \begin{tabular}{llccc}
      \hline
      Name & Usage & Density & Lower & Upper \\
      \hline
      Beta & \verb+dbetabin(a, b, n)+ &
     \multirow{2}{*}{
        $\textstyle {a+x-1 \choose x} {b+n-x-1 \choose n - x} 
                    {a+b+n-1 \choose n}^{-1}$
      } & $0$ & $n$ \\
      binomial & $a > 0, b > 0, n \in \mathbb{N}^*$ \\
      Bernoulli & \verb+dbern(p)+ & 
      \multirow{2}{*}{$p^x (1 - p)^{1 -x}$} & 
      $0$ & $1$ \\
      ~ & $0 < p < 1$ \\
      Binomial  & \verb+dbin(p,n)+ & 
      \multirow{2}{*}{${n \choose x}  p^x (1-p)^{n-x}$}
      ~  & $0$ & $n$ \\
      ~ & $0 < p < 1$, $n \in \mathbb{N}^*$ \\
      Categorical & \verb+dcat(pi)+ & \multirow{2}{*}{$\frac{\textstyle \pi_x}{\textstyle \sum_i \pi_i}$} & $1$ & $N$ \\
      ~ & $\pi \in (\mathbb{R}^+)^N$  \\
      Noncentral & \verb+dhyper(n1,n2,m1,psi)+ &
      \multirow{2}{*}{
        $\frac{ {n_1 \choose x} {n_2 \choose m_1 - x} \psi^x}
              { \sum_i {n_1 \choose i} {n_2 \choose m_1 - i} \psi^i}$
      } &
      $\scriptstyle \text{max}(0,n_+ - m_1)$ & 
      $\scriptstyle \text{min}(n_1,m_1)$ \\
      hypergeometric & $0 \leq n_i$, $0 < m_1 \leq n_+$  \\
      Negative & \verb+dnegbin(p, r)+ &
      \multirow{2}{*}{${x + r -1 \choose x} p^r (1-p)^x$} & 0 & \\
      binomial & $0 < p < 1$, $r > 0$ \\
      Poisson & \verb+dpois(lambda)+ & 
      \multirow{2}{*}{$\frac{\textstyle \exp(-\lambda) \lambda^x}{\textstyle x!}$} & 0 & \\
      ~ & $\lambda > 0$ \\
      \hline
    \end{tabular}
  \caption{Discrete univariate distributions in the \texttt{bugs} module
    \label{table:bugs:distributions:discrete}}
  \end{center}
\end{table}


\begin{table}
  \begin{center}
    \begin{tabular}{lll}
      \hline
      Name & Usage & Density \\
      \hline
      Dirichlet & \verb+p ~ ddirch(alpha)+ & 
      \multirow{2}{*}{$\Gamma(\sum_i \alpha_i) \prod_j 
        \frac{\textstyle p_j^{\alpha_j - 1}}{\textstyle \Gamma(\alpha_j)}$} \\
      ~ & $\alpha_j \geq 0$ \\
      & \\
      Multivariate & \verb+x ~ dmnorm(mu,Omega)+ &
      \multirow{2}{*}{
        $\left(\frac{|\Omega|}{2\pi}\right)^{\frac{1}{2}} exp\{-(x-\mu)^T \Omega (x-\mu) / 2\}$} \\
      normal & $\Omega$ positive definite \\
      Wishart & \verb+Omega ~ dwish(R,k)+ &
      \multirow{2}{*}{
        $\frac{\textstyle |\Omega|^{(k-p-1)/2} |R|^{k/2} \exp\{-\text{Tr}(R\Omega/2)\}}
               {\textstyle 2^{pk/2} \Gamma_p (k/2)}$
      } \\
      & $R \; p \times p$ pos. def., $k \geq p$ \\
      Multivariate & \verb+x ~ dmt(mu,Omega,k)+ &
      \multirow{2}{*}{
        $\frac{\textstyle \Gamma \{(k+p)/2\}}{\textstyle \Gamma(k/2) (n\pi)^{p/2}}
        |\Omega|^{1/2}
        \left\{1 + \frac{1}{k} (x - \mu)^T \Omega (x - \mu) \right\}^{-\frac{(k+p)}{2}}$   } \\
      Student t &  $\Omega$ pos. def. & \\
      Multinomial  & \verb+x ~ dmulti(pi, n)+ & 
      \multirow{2}{*}{$n! \prod_j 
        \frac{\textstyle \pi_j^{x_j}}{\textstyle x_j!}$} \\
      ~ & $\sum_j x_j = n$ \\
      & \\
    \hline
    \end{tabular}
    \caption{Multivariate distributions in the \texttt{bugs} module
      \label{table:bugs:distributions:multi}}
  \end{center}
\end{table}

\section{Distribution aliases}
\label{subsection:distributions:aliases}

A distribution may optionally have an alias, which can be used in the
model definition in place of the canonical name. Aliases are used to
to avoid confusion with other statistical software in which
distributions may have different names. Table
\ref{table:bugs:distributions:alias} shows the distributions in the
\texttt{bugs} module with an alias.

\begin{table}
\begin{center}
\begin{tabular}{llll}
\hline
Distribution & Canonical & Alias & Compatibile  \\
             & name      &       & with         \\
\hline
Binomial           & dbin      & dbinom   & R   \\
Chi-square         & dchisqr   & dchisq   & R   \\ 
Negative binomial  & dnegbin   & dnbinom  & R   \\
Weibull            & dweib     & dweibull & R   \\ 
Dirichlet          & ddirch    & ddirich  & OpenBUGS \\
\hline
\end{tabular}
\caption{Distributions with aliases in \texttt{bugs} module
  \label{table:bugs:distributions:alias}}
\end{center}
\end{table}

\chapter{Differences between \JAGS\ and \OpenBUGS}

Although \JAGS\ aims for the same functionality as \OpenBUGS, there are
a number of important differences.

\subsection{Data format}

There is no need to transpose matrices and arrays when transferring
data between \R\ and \JAGS, since \JAGS\ stores the values of an array
in ``column major'' order, like \R\ and FORTRAN ({\em i.e.} filling
the left-hand index first).

If you have an \textsf{S}-style data file for \OpenBUGS\ and you wish
to convert it for \JAGS, then use the command \texttt{bugs2jags},
which is supplied with the \CODA\ package.

\subsection{Distributions}

Structural zeros are allowed in the Dirichlet distribution. If
\begin{verbatim}
p ~ ddirch(alpha)
\end{verbatim}
and some of the elements of alpha are zero, then the corresponding
elements of p will be fixed to zero.

The Multinomial (\verb+dmulti+) and Categorical (\verb+dcat+)
distributions, which take a vector of probabilities as a parameter,
may use unnormalized probabilities. The probability vector is
normalized internally so that
\[
p_i \rightarrow \frac{p_i}{\sum_j p_j}
\]

The non-central hypergeometric distribution (\verb+dhyper+) uses the
same parameterization as R, which is different from the
parameterization used in OpenBUGS 3.2.2. OpenBUGS is parameterized as
\begin{verbatim}
X ~ dhyper(n, m, N, psi)     #OpenBUGS
\end{verbatim}
where $n, m, N$ are the following table margins:
\begin{center}
\begin{tabular}{|cc|c|}
\hline
x & - & n \\
-  & - & -  \\
\hline
m & - & N \\
\hline
\end{tabular}
\end{center}
This parameterization is symmetric in $n$, $m$. In JAGS, \verb+dhyper+
is parameterized as
\begin{verbatim}
X ~ dhyper(n1, n2, m1, psi) #JAGS 
\end{verbatim}
where $n1, n2, m1$ are
\begin{center}
\begin{tabular}{|cc|c|}
\hline
x & - & m1 \\
-  & - & -   \\
\hline
n1 & n2 & - \\
\hline
\end{tabular}
\end{center}

\subsection{Observable Functions}
\label{section:obfun}

Logical nodes in the \BUGS\ language are a convenient way of
describing the relationships between observables (constant and
stochastic nodes), but are not themselves observable. You cannot
supply data values for a logical node.  

This restriction can occasionally be inconvenient, as there are
important cases where the data are a deterministic function of
unobserved variables.  Three important examples are
\begin{enumerate}
\item Censored data, which commonly occurs in survival analysis. In
the most general case, we know that unobserved failure time $T$
lies in the interval $(L,U]$.
\item Rounded data, when there is a discrete underlying distribution
but the measurements are rounded to a fixed number of decimal places.
\item Aggregate data when we observe the sum of two or more
unobserved variables.
\end{enumerate}
\JAGS\ contains novel distributions to handle these situations.  

\subsubsection{Interval censored data: \texttt{dinterval}}

\begin{verbatim}
t <- c(1.2, 4.7, NA, NA, 3.2)
is.censored <- c(0, 0, 1, 1, 0)
cutpoint <- 5
\end{verbatim}

The \texttt{dinterval} distribution represents interval-censored
data. It has two parameters: $t$ the original continuous variable, and
$c[]$, a vector of cut points of length $M$, say. If \texttt{Y $\sim$
  dinterval(t, c)} then

\begin{tabular}{lll}
$Y=0$   & if & $t \leq c[1]$\\
$Y=m$   & if & $c[m] < t \leq c[m+1]$ for $1 \leq m < M$\\
$Y = M$ & if & $c[M] < t$.
\end{tabular}

\subsubsection{Rounded data: \texttt{dround}}

The \texttt{dround} distribution represents rounded data. It has two
parameters: $t$ the original continuous variable and $d$, the number
of decimal places to which the measurements are rounded. Thus if
$t=1.2345$ and $d=2$ then the rounded value is $1.23$. Note that $d$
can be negative: if $d=-2$ then the data are rounded to the nearest
$100$.

\subsubsection{Summed data: \texttt{dsum}}

The \texttt{dsum} distribution represents the sum of two or more
variables.  It takes a variable number of parameters. If \texttt{Y $\sim$
dsum(x1,x2,x3)} then $Y=x1+x2+x3$.

These distributions exist to give a likelihood to data that is, in fact,
a deterministic function of the parameters.  The relation
\begin{verbatim}
Y ~ dsum(x1, x2)
\end{verbatim}
is logically equivalent to
\begin{verbatim}
Y <- x1 + x2
\end{verbatim}
But the latter form does not create a contribution to the likelihood,
and does not allow you to define $Y$ as data.  The likelihood function
is trivial: it is 1 if the parameters are consistent with the data and
0 otherwise.  The \texttt{dsum} distribution also requires a special
sampler, which can currently only handle the case where the parameters
of \texttt{dsum} are unobserved stochastic nodes, and where the
parameters are either all discrete-valued or all continuous-valued. A node
cannot be subject to more than one \texttt{dsum} constraint.

\subsection{Data transformations}
\label{section:data:tranformations}

\JAGS\ allows data transformations, but the syntax is different from
\BUGS.  \BUGS\ allows you to put a stochastic node twice on the left
hand side of a relation, as in this example taken from the manual
\begin{verbatim}
   for (i in 1:N) {
      z[i] <- sqrt(y[i])
      z[i] ~ dnorm(mu, tau)
   }
\end{verbatim}
This is forbidden in \JAGS. You must put data transformations in a 
separate block of relations preceded by the keyword \texttt{data}:
\begin{verbatim}
data {
   for (i in 1:N) {
      z[i] <- sqrt(y[i])
   }
}
model {
   for (i in 1:N) {
      z[i] ~ dnorm(mu, tau)
   }
   ...
}
\end{verbatim}
This syntax preserves the declarative nature of the \BUGS\ language.
In effect, the data block defines a distinct model, which describes
how the data is generated. Each node in this model is forward-sampled
once, and then the node values are read back into the data table. The
data block is not limited to logical relations, but may also include
stochastic relations. You may therefore use it in simulations,
generating data from a stochastic model that is different from the one
used to analyse the data in the \texttt{model} statement.

This example shows a simple location-scale problem in which the ``true''
values of the parameters \texttt{mu} and \texttt{tau} are generated
from a given prior in the \texttt{data} block, and the generated
data is analyzed in the \texttt{model} block.
\begin{verbatim}
data {
   for (i in 1:N) {
      y[i] ~ dnorm(mu.true, tau.true) 
   }
   mu.true ~ dnorm(0,1);
   tau.true ~ dgamma(1,3);
}
model {
   for (i in 1:N) {
      y[i] ~ dnorm(mu, tau)
   }
   mu ~ dnorm(0, 1.0E-3)
   tau ~ dgamma(1.0E-3, 1.0E-3)
}
\end{verbatim}
Beware, however, that every node in the \texttt{data} statement will
be considered as data in the subsequent \texttt{model} statement. This
example, although superficially similar, has a quite different interpretation.
\begin{verbatim}
data {
   for (i in 1:N) {
      y[i] ~ dnorm(mu, tau) 
   }
   mu ~ dnorm(0,1);
   tau ~ dgamma(1,3);
}
model {
   for (i in 1:N) {
      y[i] ~ dnorm(mu, tau)
   }
   mu ~ dnorm(0, 1.0E-3)
   tau ~ dgamma(1.0E-3, 1.0E-3)
}
\end{verbatim}
Since the names \texttt{mu} and \texttt{tau} are used in both
\texttt{data} and \texttt{model} blocks, these nodes will be
considered as {\em observed} in the model and their values will be
fixed at those values generated in the \texttt{data} block.

\subsection{Directed cycles}

Directed cycles are forbidden in \JAGS. There are two important
instances where directed cycles are used in \BUGS.
\begin{itemize}
\item Defining autoregressive priors
\item Defining ordered priors
\end{itemize}
For the first case, the \texttt{GeoBUGS} extension to \OpenBUGS\ provides
some convenient ways of defining autoregressive priors. These should be
available in a future version of \JAGS.

\subsection{Censoring, truncation and prior ordering}
\label{section:censoring}

These are three, closely related issues that are all handled using
the \texttt{I(,)} construct in \BUGS.

Censoring occurs when a variable $X$ is not observed directly,
but is observed only to lie in the range $(L,U]$.  Censoring is
an {\em a posteriori} restriction of the data, and is represented
in OpenBUGS by the \texttt{I(,)} construct, {\em e.g.}
\begin{verbatim}
X ~ dnorm(theta, tau) I(L,U)
\end{verbatim}
where $L$ and $U$ are constant nodes.

Truncation occurs when a variable is known {\em a priori} to lie in
a certain range.  Although \BUGS\ has no construct for representing
truncated variables, it turns out that there is no difference between
censoring and truncation for top-level parameters ({\em i.e.} variables
with no unobserved parents).  Hence, for example, this
\begin{verbatim}
theta ~ dnorm(0, 1.0E-3) I(0, )
\end{verbatim}
is a perfectly valid way to describe a parameter $\theta$ with a
half-normal prior distribution.

Prior ordering occurs when a vector of nodes is known {\em a priori}
to be strictly increasing or decreasing. It can be represented in
OpenBUGS with symmetric $I(,)$ constructs,  {\em e.g.}
\begin{verbatim}
X[1] ~ dnorm(0, 1.0E-3) I(,X[2])
X[2] ~ dnorm(0, 1.0E-3) I(X[1],)
\end{verbatim}
ensures that $X[1] \leq X[2]$.

\JAGS\ makes an attempt to separate these three concepts.

Censoring is handled in \JAGS\ using the new distribution
\texttt{dinterval} (section \ref{section:obfun}). This can be
illustrated with a survival analysis example.  A right-censored
survival time $t_i$ with a Weibull distribution is described in
\OpenBUGS\ as follows:
\begin{verbatim}
t[i] ~ dweib(r, mu[i]) I(c[i], )
\end{verbatim}
where $t_i$ is unobserved if $t_i > c_i$.  In \JAGS\ this becomes
\begin{verbatim}
is.censored[i] ~ dinterval(t[i], c[i])
t[i] ~ dweib(r, mu[i])
\end{verbatim}
where \verb+is.censored[i]+ is an indicator variable that takes the
value 1 if $t_i$ is censored and 0 otherwise. See the MICE and KIDNEY
examples in the ``classic bugs'' set of examples.

Truncation is represented in \JAGS\ using the \texttt{T(,)} construct,
which has the same syntax as the \texttt{I(,)} construct in \OpenBUGS,
but has a different interpretation. If
\begin{verbatim}
X ~ dfoo(theta) T(L,U)
\end{verbatim}
then {\em a priori} $X$ is known to lie between $L$ and $U$. This
generates a likelihood
\[
\frac{p(x \mid \theta)}{P(L \leq X \leq U \mid \theta)}
\]
if $L \leq X \leq U$ and zero otherwise, where $p(x \mid \theta)$ is
the density of $X$ given $\theta$ according to the distribution
\texttt{foo}. Note that calculation of the denominator may be
computationally expensive.

For compatibility with OpenBUGS, JAGS permits the use of \texttt{I(,)}
for truncation when the the parameters of the truncated distribution
are fixed.  For example, this is permitted:
\begin{verbatim}
mu ~ dnorm(0, 1.0E-3) I(0, )
\end{verbatim}
because the truncated distribution has fixed parameters (mean 0,
precision 1.0E-3).  In this case, there is no difference between 
censoring and truncation.  Conversely, this is not permitted:
\begin{verbatim}
for (i in 1:N) {
   x[i] ~ dnorm(mu, tau) I(0, )
}
mu ~ dnorm(0, 1.0E-3)
tau ~ dgamma(1, 1)
}
\end{verbatim}
because the mean and precision of $x_1 \dots x_N$ are parameters to be
estimated.  JAGS does not know if the aim is to model truncation or censoring
and so the compiler will reject the model. Use either \texttt{T(,)} or the
\texttt{dinterval} distribution to resolve the ambiguity.

Prior ordering of top-level parameters in the model can be achieved
using the \texttt{sort} function, which sorts a vector in ascending
order.

Symmetric truncation relations like this
\begin{verbatim}
alpha[1] ~ dnorm(0, 1.0E-3) I(,alpha[2])
alpha[2] ~ dnorm(0, 1.0E-3) I(alpha[1],alpha[3])
alpha[3] ~ dnorm(0, 1.0E-3) I(alpha[2],)
\end{verbatim}
Should be replaced by this
\begin{verbatim}
for (i in 1:3) {
   alpha0[i] ~ dnorm(0, 1.0E-3)
}
alpha[1:3] <- sort(alpha0)
\end{verbatim}

\chapter{Feedback}

Please send feedback to \url{martyn_plummer@users.sourceforge.net}.
I am particularly interested in the following problems:

\begin{itemize}
\item Crashes, including both segmentation faults and uncaught exceptions.
\item Incomprehensible error messages
\item Models that should compile, but don't 
\item Output that cannot be validated against \OpenBUGS
\item Documentation erors
\end{itemize}

If you want to send a bug report, it must be reproducible. Send the
model file, the data file, the initial value file and a script file
that will reproduce the problem. Describe what you think should
happen, and what did happen.

\chapter{Acknowledgments}

Many thanks to the \BUGS\ development team, without whom \JAGS\ would
not exist.  Thanks also to Simon Frost for pioneering \JAGS\ on
Windows and Bill Northcott for getting \JAGS\ on Mac OS X to
work. Kostas Oikonomou found many bugs while getting \JAGS\ to work on
Solaris using Sun development tools and libraries.  Bettina Gruen,
Chris Jackson, Greg Ridgeway and Geoff Evans also provided useful
feedback.  Special thanks to Jean-Baptiste Denis who has been very
diligent in providing feedback on JAGS.

\appendix
\chapter{Data format}
\label{appendix:data}

The file format used by JAGS for representing data and initial values
is the \verb+dump()+ format used by R.  This format is valid R code,
so a JAGS data file can be read into R using the \verb+source()+ function.
Since R is a functional language, the code consists of a sequence
of assignments and function calls.
\begin{description}
\item[Assignments] are represented by a left arrow (\verb+<-+) with
  the variable name on the left hand side and a numeric value or
  function call on the right hand side. The variable name may
  optionally be enclosed in single quotes, double quotes, or back-ticks.
\item[Function calls] are represented by the function name, followed
  by a list of comma-separated arguments inside round brackets.
  Optional arguments need to be tagged (i.e. given in the form
  tag=value).
\end{description}
The \JAGS\ parser ignores all white space.  Long expressions can
therefore be split over several lines.

Scalar values are represented by a simple assignment statement 
\begin{verbatim}
theta <- 0.1243
\end{verbatim}
All numeric values are read in to JAGS as doubles, even if they are
represented as integers (i.e. `12' and `12.0' are
equivalent). Engineering notation may be used to represent large or
small values (e.g 1.5e-3 is equivalent to 0.0015).

Vectors are denoted by the \R\ collection function ``c'', which takes
a number of arguments equal to the length of the vector.  The
following code denotes a vector of length 4:
\begin{verbatim}
x <- c(2, 3.5, 1.3e-2, 88)
\end{verbatim}
There is no distinction between row and column-vectors. 

Matrices and higher-dimensional arrays in R are created by adding a
dimension attribute (\verb+.Dim+) to a vector. In the R dump format
this is done using the ``structure'' function.  The first argument to
the structure function is a vector of numeric values of the matrix,
given in column major order (i.e. filling the left index first). The
second argument, which must be given the tag \verb+.Dim+, is the
number of dimensions of the array, represented by a vector of integer
values. For example, if the matrix ``A'' takes values
\[
\left(
\begin{array}{cc}
  1 & 4 \\
  2 & 5 \\
  3 & 6 
\end{array}
\right)
\]
it is represented as
\begin{verbatim}
`A` <- structure(c(1, 2, 3, 4, 5, 6), .Dim=c(3,2))
\end{verbatim}

The simplest way to prepare your data is to read them into \R\ and
then dump them.  Only numeric vectors, matrices and arrays are
allowed. More complex data structures such as factors, lists and data
frames cannot be parsed by \JAGS\, nor can non-numeric vectors.  Any
\R\ attributes of the data (such as names and dimnames) are ignored
when they are read into \JAGS. 

\bibliographystyle{plain}
\bibliography{jags_user_manual}

\end{document}



