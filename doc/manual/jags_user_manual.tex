\documentclass[11pt, a4paper, titlepage]{report}
\usepackage{amsmath}
\usepackage{a4wide}
\usepackage{url}
\usepackage{multirow}
\usepackage{amsfonts}
\newcommand{\release}{2.0.0}
\newcommand{\JAGS}{\textsf{JAGS}}
\newcommand{\BUGS}{\textsf{BUGS}}
\newcommand{\WinBUGS}{\textsf{WinBUGS}}
\newcommand{\R}{\textsf{R}}
\newcommand{\CODA}{\textsf{coda}}
\begin{document}

\title{JAGS Version \release\ manual}
\author{Martyn Plummer}
\maketitle

\tableofcontents

\chapter{Introduction}

JAGS is Just Another Gibbs Sampler.  It is a program for the analysis
of Bayesian models using Markov Chain Monte Carlo (MCMC) which is not
wholly unlike
\WinBUGS\ (\url{http://www.mrc-bsu.cam.ac.uk}). \JAGS\ was written
with three aims in mind: to have an engine for the \BUGS\ language
that runs on Unix; to be extensible, allowing users to write their own
functions, distributions, and samplers; and to be a platform for
experimentation with ideas in Bayesian modelling.

\JAGS\ is designed to work closely with the \R\ language and
environment for statistical computation and graphics
(\url{http://www.r-project.org}).  In particular, you will need \R\ to
prepare the input data for \JAGS, and you will find it useful to
install the \CODA\ package for \R\ to analyze the output.

\JAGS\ is licensed under the GNU General Public License
version 2. You may freely modify and redistribute it under certain
conditions (see the file \texttt{COPYING} for details).

\chapter{Running a model in \JAGS}

\JAGS\ is designed for inference on Bayesian models using Markov Chain
Monte Carlo (MCMC) simulation.  Running a model refers to generating
samples from the posterior distribution of the model parameters.  This
takes place in five steps:
\begin{enumerate}
\item Defining the model
\item Compiling
\item Initializing
\item Adapting and burning-in
\item Monitoring
\end{enumerate}
The next stages of analysis are done outside of \JAGS: convergence
diagnostics, model criticism, and summarizing the samples must be done
using other packages more suited to this task. There are several
\R\ packages designed for analyzing MCMC output, and \JAGS\ is
designed for easy transition of data to and from \R.

\section{Definition}

There are two parts to the definition of a model in \JAGS: a
description of the model and the definition of the data.

\subsection{Model definition}

The model is defined in a text file using a dialect of the
\BUGS\ language.  The model definition consists of a series of
relations inside a block delimited by curly brackets \verb+{+ and
\verb+}+ and preceded by the keyword \verb+model+. Here is the standard
linear regression example:

\begin{verbatim}
model {
    for (i in 1:N) {
          Y[i]   ~ dnorm(mu[i], tau)
          mu[i] <- alpha + beta * (x[i] - x.bar)
    }
    x.bar   <- mean(x)
    alpha    ~ dnorm(0.0, 1.0E-4)
    beta     ~ dnorm(0.0, 1.0E-4)
    sigma   <- 1.0/sqrt(tau)
    tau      ~ dgamma(1.0E-3, 1.0E-3)
}
\end{verbatim}

Each relation defines a node in the model in terms of other nodes that
appear on the right hand side. These are referred to as the parent
nodes. Taken together, the nodes in the model (together with the
parent/child relationships represented as directed edges) form a
directed acyclic graph. The very top-level nodes in the graph, with no
parents, are constant nodes, which are defined either in the model
definition ({\em e.g.}  \verb+1.0E-3+), or in the data file ({\em
  e.g.}  \verb+x[1]+).

Relations can be of two types. A {\em stochastic relation} (\verb+~+)
defines a stochastic node, representing a random variable in the
model. A {\em deterministic relation} (\verb+<-+) defines a
deterministic node, the value of which is determined exactly by the
values of its parents.

Nodes defined by a relation are embedded in named arrays. Array names may
contain letters, numbers, decimal points and underscores, but they must
start with a letter.  The node array \verb+mu+ is a vector of length
$N$ containing $N$ nodes (\verb+mu[1]+, $\ldots$, \verb+mu[N]+). The
node array \verb+alpha+ is a scalar.  \JAGS\ follows the S language
convention that scalars are considered as vectors of length 1. Hence
the array \verb+alpha+ contains a single node \verb+alpha[1]+.

Deterministic nodes do not need to be embedded in node arrays. The
node \verb+Y[i]+ could equivalently be defined as
\begin{verbatim}
Y[i] ~ dnorm(alpha + beta * (x[i] - x.bar), tau)
\end{verbatim}
In this version of the model definition, the node previously defined
as \verb+mu[i]+ still exists, but is not accessible to the user as it
does not have a name.  This ability to hide deterministic nodes by
embedding them in other expressions underscores an important fact:
only the stochastic nodes in a model are really
important. Deterministic nodes are merely a syntactically convenient
way of describing the relations between, or transformations of, the
stochastic nodes.

\subsection{Data}
\label{section:data}

The data are defined in a separate file from the model definition, in
the format created by the \texttt{dump()} function in \R.  The
simplest way to prepare your data is to read them into \R\ and then
dump them.  Only numeric vectors, matrices and arrays are
allowed. More complex data structures such as factors, lists and data
frames cannot be parsed by \JAGS\, nor can non-numeric vectors.  Any
\R\ attributes of the data (such as names and dimnames) are stripped
when they are read into \JAGS.

The data may contain missing values, but you cannot supply partially
missing values for a multivariate node. In \JAGS\ a node is either
completely observed, or completely unobserved.  The unobserved nodes
are referred to as the {\em parameters} of the model. The data file
therefore defines, by omission, the parameters of the model.

Here are the data for the \verb+LINE+ example:
\begin{verbatim}
`x` <-
c(1, 2, 3, 4, 5)
#R-style comments, like this one, can be embedded in the data file
`Y` <-
c(1, 3, 3, 3, 5)
`N` <-
5
\end{verbatim}

It is an error to supply a data value for a deterministic node. (See, however,
section \ref{section:obfun} on observable functions).

\subsection{Node Array dimensions}

\subsubsection*{Array declarations}

\JAGS\ allows the option of declaring the dimensions of node arrays in
the model file. The declarations are modelled on classic \BUGS, and
consist of the keyword \texttt{var} (for variable) followed by a
comma-separated list of array names, with their dimensions in
square brackets. The dimensions may be given in terms of any
expression of the data that returns a single integer value.

In the linear regression example, the model block could be preceded by
\begin{verbatim}
var x[N], Y[N], mu[N], alpha, beta, tau, sigma, x.bar;
\end{verbatim}

\subsubsection*{Undeclared nodes}

If a node array is not declared then JAGS has three methods of
determining its size.
\begin{enumerate}
\item {\bf Using the data.}  The dimension of an undeclared node array
  may be inferred if it is supplied in the data file.
\item {\bf Using the left hand side of the relations.}  The maximal
  index values on the left hand side of a relation are taken to be the
  dimensions of the node array.  For example, in this case:
\begin{verbatim}
for (i in 1:N) {
   for (j in 1:M) {
      Y[i,j] ~ dnorm(mu[i,j], tau)
   }
}
\end{verbatim}
$Y$ would be inferred to be an $N \times M$ matrix. Using this method,
empty indices are not allowed on the left hand side of any relation.
\item {\bf Using the dimensions of the parents} If a whole node array
  appears on the left hand side of a relation, then its dimensions can
  be inferred from the dimensions of the nodes on the right hand side.
  For example, if \verb+A+ is known to be an $N \times N$ matrix
  and
\begin{verbatim}
B <- inverse(A)
\end{verbatim}
Then \verb+B+ is also an $N \times N$ matrix.
\end{enumerate}

\subsubsection*{Querying array dimensions}  

The \JAGS\ compiler has two built-in functions for querying array
sizes.  The \verb+length()+ function returns the number of elements in
a node array, and the \verb+dim()+ function returns a vector
containing the dimensions of an array.  These two functions may be
used to simplify the data preparation. For example, if \verb+Y+
represents a vector of observed values, then using the \verb+length()+
function in a for loop:
\begin{verbatim}
for (i in 1:length(Y)) {
    Y[i] ~ dnorm(mu[i], tau)
}
\end{verbatim}
avoids the need to put a separate data value \verb+N+ in the file
representing the length of \verb+Y+.  

For multi-dimensional arrays, the \verb+dim+ function serves a similar
purpose. The \verb+dim+ function returns a vector, which must be stored
in an array before its elements can be accessed. For this reason, calls
to the \verb+dim+ function must always be in a data block (see section
\ref{section:data:tranformations}).
\begin{verbatim}
data {
   D <- dim(Z)
}
model {
   for (i in 1:D[1]) {
      for (j in 1:D[2]) {
         Z[i,j] <- dnorm(alpha[i] + beta[j], tau)
      }
   }
   ...
}
\end{verbatim}
Clearly, the \verb+length()+ and \verb+dim()+ functions can only
work if the size of the node array can be inferred, using one of the
three methods outlined above.

Note: the \verb+length()+ and \verb+dim()+ functions are different
from all other functions in \JAGS: they do not act on nodes, but only
on node {\em arrays}. As a consequence, an expression such as
\verb+dim(a %*% b)+ is syntactically incorrect.

\section{Compilation}

When a model is compiled, a graph representing the model is created in
computer memory. Compilation can fail for a number of reasons:
\begin{enumerate}
\item The graph contains a directed cycle.  These are forbidden
in \JAGS.
\item A top-level parameter is undefined. Any node that is used on
the right hand side of a relation, but is not defined on the left
hand side of any relation, is assumed to be a constant node. Its value
must be supplied in the data file. 
\item The model uses a function or distribution that has not been
defined in any of the loaded modules.
\end{enumerate}
The number of parallel chains to be run by \JAGS\ is also defined at
compilation time.  Each parallel chain should produce an independent
sequence of samples from the posterior distribution. By default,
\JAGS\ only runs a single chain.

\section{Initialization}

Before a model can be run, it must be initialized. There are three
steps in the initialization of a model:
\begin{enumerate}
\item The initial values of the model parameters are set.
\item A Random Number Generator (RNG) is chosen for each parallel chain,
  and its initial value is set.
\item The Samplers are chosen for each parameter in the model. 
\end{enumerate}

\subsection{Parameter values}

The user may supply an initial value file containing values for the
model parameters. The file may not contain values for logical or
constant nodes. The format is the same as the data file (see section
\ref{section:data}).

If initial values are not supplied by the user, then each parameter
chooses its own initial value based on the values of its parents.  The
method of choosing the initial value depends on whether the node is
classified as a fixed effect or a random effect.\footnote{Jean
  Baptiste Denis points out that this distinction between fixed and
  random effects is a little artificial since it depends on the
  parameterization of the model.  However, within the context of a
  single parameterization, defined by a particular BUGS-language
  representation of the model, the distinction is unambiguous.}
\begin{description}
\item[Fixed effects] are parameters for which all parents are
  observed.  In this case, the initial value is chosen to be a
  ``typical value'' from the prior distribution. The exact meaning of
  ``typical value'' depends on the distribution of the stochastic
  node, but is usually the mean, median, or mode.
\item[Random effects] are parameters for which one or more parents are
  unobserved.  The initial values of these nodes are chosen by taking
  a random sample from the prior distribution
\end{description}
These heuristic rules are designed to create reasonable default starting
values in a wide range of models.

If you rely on automatic initial value generation and are running
multiple parallel chains, then the values for the top-level model
parameters will be the same in all chains.  You may not want this
behaviour, especially if you are using the Gelman and Rubin convergence
diagnostic, which assumes that the initial values are over-dispersed
with respect to the posterior distribution. In this case, you are
advised to set the starting values manually using the "parameters in"
statement.

\subsection{RNGs}
\label{section:rngs}

Each chain in \JAGS\ has its own random number generator (RNG). RNGs
are more correctly referred to as {\em pseudo}-random number
generators. They generate a sequence of numbers that merely looks
random but is, in fact, entirely determined by the initial state.  You
may optionally set the name of the RNG and its initial state in the
initial values file.

The name of the RNG is set as follows. 
\begin{verbatim}
.RNG.name <- "name"
\end{verbatim}
There are four RNGs supplied by the \texttt{base} module in \JAGS\
with the following names:
\begin{verbatim}
"base::Wichmann-Hill"
"base::Marsaglia-Multicarry"
"base::Super-Duper"
"base::Mersenne-Twister"
\end{verbatim}

There are two ways to set the starting state of the RNG. The simplest
is to supply an integer value to \texttt{.RNG.seed}, {\em e.g.}
\begin{verbatim}
".RNG.seed" <- 314159
\end{verbatim}
The second is way to save the state of the RNG from one JAGS session
(see the ``PARAMETERS TO'' statement, section \ref{parameters:to}) and
use this as the initial state of a new chain. The state of any RNG in
JAGS can be saved and loaded as an integer vector with the name
\texttt{.RNG.state}. For example,
\begin{verbatim}
".RNG.state" <- as.integer(c(20899,10892,29018))
\end{verbatim}
is a valid state for the Marsaglia-Multicarry generator.  You cannot
supply an arbitrary integer to \texttt{.RNG.state}. Both the length of
the vector and the permitted values of its elements are determined by
the class of the RNG. The only safe way to use \texttt{.RNG.state} is
to re-use a previously saved state.

If no RNG names are supplied, then RNGs will be chosen automatically
so that each chain has its own independent random number stream.  The
exact behaviour depends on which modules are loaded. The
\texttt{base} module uses the four generators listed above for the
first four chains.  If you want to work with more than four parallel
chains, then you need to explicitly supply the RNG name and the random
seed for each chain.

By default, \JAGS\ bases the initial state on the time stamp. This
means that, when a model is re-run, it generates an independent set of
samples. If you want your model run to be reproducible, you must
explicitly set the \verb+.RNG.seed+ for each chain.

\subsection{Samplers}

A Sampler is an object that acts on a set of parameters and updates
them from one iteration to the next. During initialization of the
model, Samplers are chosen automatically for all parameters. 

The Model holds an internal list of {\em Sampler Factory} objects,
which inspect the graph, recognize sets of parameters that can be
updated with specific methods, and generate Sampler objects for
them. The list of Sampler Factories is traversed in order, starting with
sampling methods that are efficient, but limited to certain specific
model structures and ending with the most generic, possibly
inefficient, methods. If no suitable Sampler can be generated for one
of the model parameters, an error message is generated.

The user has no direct control over the process of choosing
Samplers. However, you may indirectly control the process by loading a
module that defines a new Sampler Factory. The module will insert the
new Sampler Factory at the beginning of the list, where it will be
queried before all of the other Sampler Factories.

A report on the samplers chosen by the model, and the stochastic nodes
they act on, can be generated using the ``SAMPLERS TO'' command. See 
section \ref{samplers:to}.
 
\section{Adaptation and burn-in}

In theory, output from an MCMC sampler converges to the target
distribution ({\em i.e.} the posterior distribution of the model
parameters) in the limit as the number of iterations tends to
infinity. In practice, all MCMC runs are finite.  By convention, the
MCMC output is divided into two parts: an initial ``burn-in'' period,
which is discarded, and the remainder of the run, in which the output
is considered to have converged (sufficiently close) to the target
distribution. Samples from the second part are used to create
approximate summary statistics for the target distribution.

By default, \JAGS\ keeps only the current value of each node in the
model, unless a monitor has been defined for that node. The burn-in
period of a \JAGS\ run is therefore the interval between model
initialization and the creation of the first monitor.

When a model is initialized, it is in {\em adaptive mode}, meaning
that the Samplers used by the model may modify their behaviour for
increased efficiency. Since this adaptation may depend on the entire
sample history, the sequence generated by an adapting sampler is no
longer a Markov chain, and is not guaranteed to converge to the target
distribution. Therefore, adaptive mode must be turned off at some
point during burn-in, and a sufficient number of iterations must take
place {\em after} the adaptive phase to ensure convergence.

By default, adaptive mode is turned off half way through first update
of a \JAGS\ model, although the user may also control the length of
the adaptive phase directly.  All samplers have a built in test to
determine whether they have converged to their optimal sampling
behaviour.  If any sampler fails this validation test, a warning will
be printed. To ensure optimal sampling behaviour, the model should be
run again from scratch using a longer adaptation period.

\section{Monitoring}
\label{section:monitoring}

A {\em monitor} in \JAGS\ is an object that records sampled
values. The simplest monitor is a {\em trace monitor}, which stores
the sampled value of a node, and can dump the stored values to file
in the format used by the CODA package for \R\ and S-PLUS.

More complex monitors can be defined.  For example, the \verb+dic+
module defines a {\em deviance monitor}, which records the deviance of
observed stochastic nodes, and a {\em pD monitor}, which estimates the
contribution of a stochastic node to the {\em effective number of
parameters} ($p_D$) of a model.

All monitors can be dumped to a file in \R\ \verb+dump()+ format in
the form of a list.

\chapter{Running \JAGS}

\JAGS\ has a command line interface. To invoke jags interactively,
simply type \texttt{jags} at the shell prompt on Unix, or the Windows
command prompt on Windows. To invoke JAGS with a script file, type
\begin{verbatim}
jags <script file>
\end{verbatim}
Output from \JAGS\ is printed to the standard output, even when a
script file is being used.  The \JAGS\ interface is designed to be
forgiving. It will print a warning message if you make a mistake, but
otherwise try to keep going.  This may create a cascade of error messages,
of which only the first is informative.

\section{Scripting commands}
\label{section:scripting}

\JAGS\ has a simple set of scripting commands with a syntax loosely
based on \textsf{Stata}. Commands are shown below preceded by a dot
(.). This is the \JAGS\ prompt. Do not type the dot in when you are
entering the commands.

C-style block comments taking the form /* ... */ can be
embedded anywhere in the script file.  Additionally, you may use
\R-style single-line comments starting with \#.

If a scripting command takes a file name, then the name may be
optionally enclosed in quotes. Quotes are required when the file name
contains space, or any character which is not alphanumeric, or one of
the following: \verb+_+, \verb+-+, \verb+.+, \verb+/+, \verb+\+.

In the descriptions below, angular brackets \verb+<>+, and the text
inside them, represents a parameter that should be replaced with the
correct value by you.  Anything inside square brackets \verb+[]+ is
optional. Do not type the square brackets if you wish to use an
option.

\subsection{MODEL IN}

\begin{verbatim}
. model in <file>
\end{verbatim}
Checks the syntactic correctness of the model description in
\texttt{file} and reads it into memory. The next compilation
statement will compile this model. 

See also: MODEL CLEAR (\ref{model:clear})

\subsection{DATA IN}
\label{data:in}

\begin{verbatim}
. data in <file>
\end{verbatim}
JAGS keeps an internal data table containing the values of observed
nodes inside each node array.  The DATA IN statement reads data from a
file into this data table.

Several data statements may be used to read in data from more than one
file. If two data files contain data for the same node array, the second
set of values will overwrite the first, and a warning will be printed.

See also: DATA TO (\ref{data:to}).

\subsection{COMPILE}

\begin{verbatim}
. compile [, nchains(<n>)]
\end{verbatim}
Compiles the model using the information provided in the preceding
model and data statements. By default, a single Markov chain is
created for the model, but if the \texttt{nchains} option is given,
then \texttt{n} chains are created 

Following the compilation of the model, further DATA IN statements are
legal, but have no effect.  A new model statement, on the other hand,
will replace the current model.

\subsection{PARAMETERS IN}
\label{parameters:in}

\begin{verbatim}
. parameters in <file> [, chain(<n>)]
\end{verbatim}
Reads the values in \texttt{file} and writes them to the corresponding
parameters in chain \texttt{n}. The file has the same format as the
one in the DATA IN statement.  The \texttt{chain} option may be
omitted, in which case the parameter values in all chains are set to
the same value.

The PARAMETERS IN statement may be used before a model has been
initialized.  You may only supply the values of unobserved stochastic
nodes in the parameters file. Logical nodes and constant nodes are
forbidden.

See also: PARAMETERS TO (\ref{parameters:to})

\subsection{INITIALIZE}

\begin{verbatim}
. initialize
\end{verbatim}
Initializes the model using the previously supplied data and parameter
values supplied for each chain.

\subsection{UPDATE}

\begin{verbatim}
. update <n> [,by(<m>)]
\end{verbatim}
Updates the model by \texttt{n} iterations. 

The first UPDATE statement turns off adaptive mode for all samplers in
the model after \texttt{n/2} iterations. A warning is printed 
if adaptation is incomplete. In this case the model must be run again
with a longer adaptation phase, starting from the MODEL IN statement.

A progress bar is printed on the standard output consisting of 50
asterisks. If the \texttt{by} option is supplied, a new asterisk is
printed every \texttt{m} iterations. If this entails more than 50
asterisks, the progress bar will be wrapped over several lines.  If
\texttt{m} is zero, the printing of the progress bar is suppressed.

\subsection{ADAPT}

\begin{verbatim}
. adapt <n> [,by(<m>)]
\end{verbatim}
Updates the model by \texttt{n} iterations and then turns of
adaptive mode.

Use this instead of the first UPDATE statement if you want explicit
control over the length of the adaptive sampling phase.

\subsection{MONITOR}
\label{section:monitor}

In \JAGS, a monitor is an object that records a value based on the
current state of a node.  The simplest monitor simply records the
value of the node itself. This is called a ``trace'' monitor. More
complex monitors can be defined that do additional calculations.  For
example, the \verb+dic+ module defines a ``deviance'' monitor that
records the deviance of an observed stochastic node, and a ``pD''
monitor that estimates the contribution of a node to the effective
number of parameters in the model.

In order to distinguish between different classes of Monitor, all
monitor statements in the \JAGS\ scripting language have a
\texttt{type} option. If this is omitted, a trace monitor is assumed.

\begin{verbatim}
. monitor <varname> [, thin(n), type(<name>)]
\end{verbatim}
Sets a monitor for variable \texttt{<varname>}, The \texttt{thin}
option sets the thinning interval of the monitor so that it will only
record every nth value. 

\begin{verbatim}
. monitor [, type(<name>)]
\end{verbatim}
A monitor keyword on its own, with no variable name, will create a set
of monitors for default nodes in the model.  The nodes chosen as
``default'' nodes depend on the type of monitor.  For a trace monitor,
the default nodes are all fixed effects ({\em i.e.} all model
parameters with observed parents).

\begin{verbatim}
. monitor clear <varname> [, type(<name>)]
\end{verbatim}
Clears the monitor of the given type associated with variable
\texttt{<varname>}. 

\begin{verbatim}
. monitors to <filename> [, type(<name>)]
\end{verbatim}
Dumps the values of all monitors of the given type to the given file
in a format that can be read into \R\ using the \verb+source()+
function.  The \R\ structure is a list with one entry for each sampler.

Note the plural (``monitors to'' not ``monitor to'').

\subsection{CODA}

\begin{verbatim}
. coda <varname> [, stem(<filename>)] 
\end{verbatim}
If the named node has a trace monitor, this dumps the monitored values
of to files \texttt{CODAindex.txt}, \texttt{CODAindex1.out},
\texttt{CODAindex2.txt}, \ldots in a form that can be read by the
\CODA\ package of \R.  The file name stem may be changed from
\texttt{CODA} to another value by using the \texttt{stem()}
option. The wild-card character ``*'' may be used to dump all
monitored nodes

\subsection{EXIT}

\begin{verbatim}
. exit
\end{verbatim}
Exits \JAGS. \JAGS\ will also exit when it reads an end-of-file character.
Note that although \JAGS\ behaves in many ways like classic \BUGS, it 
will not automatically dump the contents of the monitors on exit. You
must use the CODA statement before exiting.

\subsection{DATA TO}
\label{data:to}
\begin{verbatim}
. data to <filename>
\end{verbatim}
Writes the data ({\em i.e.} the values of the observed nodes) to a
file in the R \texttt{dump} format. The same file can be used in a
DATA IN statement for a subsequent model.

See also: DATA IN (\ref{data:in})

\subsection{PARAMETERS TO}
\label{parameters:to}
\begin{verbatim}
. parameters to <file> [, chain(<n>)]
\end{verbatim}
Writes the current parameter values ({\em i.e.} the values of the
unobserved stochastic nodes) in chain \texttt{<n>} to a file in R dump
format. The name and current state of the RNG for chain \texttt{<n>}
is also dumped to the file.  The same file can be used as input in a
PARAMETERS IN statement in a subsequent run.

See also: PARAMETERS IN (\ref{parameters:in})

\subsection{SAMPLERS TO}
\label{samplers:to}
\begin{verbatim}
. samplers to <file>
\end{verbatim}
Writes out a summary of the samplers to the given file.  The output appears
in three tab-separated columns, with one row for each sampled node
\begin{itemize}
\item The index number of the sampler (starting with 1). The index number 
gives the order in which Samplers are updated at each iteration.
\item The name of the sampler, matching the index number
\item The name of the sampled node. 
\end{itemize}
If a Sampler updates multiple nodes then it is represented by multiple rows
with the same index number.

\subsection{LOAD}
\label{load}
\begin{verbatim}
. load <module>
\end{verbatim}
Loads a module into \JAGS\ (see chapter \ref{section:modules}). Once
loaded, a module cannot be unloaded in the same \JAGS\ session.

\subsection{MODEL CLEAR}
\label{model:clear}
\begin{verbatim}
. model clear
\end{verbatim}
Clears the current model.  The data table (see section \ref{data:in})
remains intact

\subsection{Print Working Directory (PWD)}
\begin{verbatim}
. pwd
\end{verbatim}
Prints the name of the current working directory. This is where \JAGS\
will look for files when the file name is given without a full path, 
{\em e.g.} \verb+"mymodel.bug"+.

\subsection{Change Directory (CD)}
\begin{verbatim}
. cd <dirname>
\end{verbatim}
Changes the working directory to \texttt{<dirname>}

\subsection{Directory list (DIR)}
\begin{verbatim}
. dir
\end{verbatim}
Lists the files in the current working directory.

\subsection{RUN}
\begin{verbatim}
. run <cmdfile>
\end{verbatim}
Opens the file \texttt{<cmdfile>} and reads further scripting commands
until the end of the file.  Note that if the file contains an EXIT
statement, then the \JAGS\ session will terminate. 

\section{Errors}

There are two kinds of errors in \JAGS: runtime errors, which are due to
mistakes in the model specification, and logic errors which are internal
errors in the JAGS program. 

Logic errors are generally created in the lower-level parts of the \JAGS\
library, where it is not possible to give an informative error message.
The upper layers of the \JAGS\ program are supposed to catch such errors
before they occur, and return a useful error message that will help you
diagnose the problem.  Inevitably, some errors slip through. Hence,
if you get a logic error, there is probably an error in your input to
\JAGS, although it may not be obvious what it is. Please send a bug
report (see ``Feedback'' below) whenever you get a logic error.

Error messages may also be generated when parsing files (model files,
data files, command files).  The error messages generated in this case
are created automatically by the program \texttt{yacc}. They
generally take the form ``syntax error, unexpected FOO, expecting BAR''
and are not always abundantly clear.

If a model compiles and initializes correctly, but an error occurs
during updating, then the current state of the model will be dumped
to a file named \verb+jags.dumpN.R+ where $N$ is the chain number.
You should then load the dumped data into R to inspect the state of
each chain when the error occurred.

\chapter{Modules}
\label{section:modules}

The \JAGS\  library is distributed along with certain dynamically
loadable modules that extend its functionality. A module can define
new objects of the following classes:
\begin{enumerate}
\item {\bf functions} and {\bf distributions}, the basic building
blocks of the BUGS language.
\item {\bf samplers}, the objects which update the parameters of the
model at each iteration, and {\bf sampler factories}, the objects that 
create new samplers for specific model structures.  If the module
defines a new distribution, then it will typically also define a new
sampler for that distribution.
\item {\bf monitors}, the objects that record sampled values for
later analysis, and {\bf monitor factories} that create them. 
\item {\bf random number generators}, the objects that drive the
MCMC algorithm and {\bf RNG factories} that create them.
\end{enumerate}

The \verb+base+ module and the \verb+bugs+ module are loaded automatically
at start time.  Others may be loaded by the user.

\section{The base module}

The base module supply the base functionality for the \JAGS\ library
to function correctly. It is loaded first by default.  

\subsection{Base Samplers}

The \verb+base+ module defines samplers that use highly generic update
methods.  These sampling methods only require basic information about
the stochastic nodes they sample.  Conversely, they may not be fully
efficient.

Three samplers are currently defined:
\begin{enumerate}
\item The Finite sampler can sample a discrete-valued node with
fixed support of less than 20 possible values. The node must not
be bounded using the \verb+T(,)+ construct
\item The Real Slice Sampler can sample any scalar real-valued 
stochastic node.
\item The Discrete Slice Sampler can sample any scalar
discrete-valued stochastic node.
\end{enumerate}

\subsection{Base RNGs}

The \verb+base+ module defines four RNGs, taken directly from \R,
with the following names:
\begin{enumerate}
\item \verb+"base::Wichmann-Hill"+
\item \verb+"base::Marsaglia-Multicarry"+
\item \verb+"base::Super-Duper"+
\item \verb+"base::Mersenne-Twister"+
\end{enumerate}

A single RNG factory object is also defined by the \verb+base+
module which will supply these RNGs for chains 1 to 4 respectively, if
``RNG.name'' is not specified in the initial values file.  All chains
generated by the base RNG factory are initialized using the current
time stamp.

The base RNG factory can only generate four independent RNGs from a
single seed. Hence \JAGS\ is limited to four parallel chains, unless
the user specifies the RNG name and seed in each initial value file.

\subsection{Base Monitors}

The \verb+base+ module defines the TraceMonitor class (type
``trace''). This is the monitor class that simply records the current
value of the node at each iteration.

When default nodes for monitoring are requested, the factory object
for TraceMonitors will select all fixed effects (Unobserved stochastic
nodes with observed parents) in the model. In a typical model, these
will be the highest-level model parameters.

\section{The bugs module}

The \verb+bugs+ module defines some of the functions and distributions
from \WinBUGS. These are described in more detail in sections
\ref{section:functions} and \ref{section:distributions}.  The
\verb+bugs+ module also defines conjugate samplers for efficient Gibbs
sampling.

\section{The mix module}

The \verb+mix+ module defines a novel distribution
\verb+dnormmix(mu,tau,pi)+ representing a finite mixture of normal
distributions. In the parameterization of the \verb+dnormmix+
distribution, $\mu$, $\tau$, and $\pi$ are vectors of the same length,
and the density of \verb+y ~ dnormmix(mu, tau, pi)+ is
\[
f(y | \mu, \tau, \pi) = \sum_i \pi_i \tau_i^{\frac{1}{2}} \phi( \tau^{\frac{1}{2}}_i (y - \mu_i))
\]
where $\phi()$ is the probability density function of a standard
normal distribution.

The \verb+mix+ module also defines a sampler that is designed to act
on finite normal mixtures. It uses tempered transitions to jump
between distant modes of the multi-modal posterior distribution
generated by such models. This sampler is still experimental and has
not been tested on a sufficiently wide range of problems.

\section{The dic module}

The \verb+dic+ module defines new monitor classes for Bayesian model
criticism using deviance-based measures. The \texttt{deviance} monitor
records the deviance of an observed stochastic node. The \texttt{pD}
monitor can be used to estimate the contribution of a stochastic node
to the {\em effective number of parameters} in the model. Finally, the
\texttt{popt} monitor can be used to estimate the optimism of the
expected deviance.

The \texttt{rjags} package provides classes and functions for
manipulating penalized deviances. See the rjags manual for details.
The use of these monitors through the command-line interface is
described below.

\subsection{The deviance monitor}

The \JAGS\ library already has the facility to create a deviance node
even without the \verb+dic+ module being loaded.  The deviance node
gives the {\em total} deviance of the model. The deviance monitor
defined in the \verb+dic+ module allows more a more detailed
evaluation of the model deviance by recording the deviance of {\em
  individual} nodes. The command
\begin{verbatim}
monitor, type(deviance)
\end{verbatim}
will create a deviance monitor for all observed stochastic
nodes. These monitors cannot be dumped to file in CODA format. Instead
they are dumped using the command
\begin{verbatim}
monitors to "myfile.R", type(deviance)
\end{verbatim}
The file ``myfile.R'' can then be read into \R\ with the \texttt{source()}
function. This creates a list named ``deviance'' which contains the
sampled values.

\subsection{The \texttt{pD} monitor}

The \verb+pD+ monitor estimates the contribution to the effective
number of parameters ($p_D$) \cite{spiegelhalter:etal:2002} from an
observed stochastic node by comparing the deviance deviance across
multiple chains \cite{plummer:2002}. It is created by using the option
\texttt{type(pD)}. If the model has only one chain then a \verb+pD+
monitor cannot be defined.

Like the deviance monitor, the \verb+pD+ monitor factory will create a
monitor for every observed stochastic node using the command.
\begin{verbatim}
monitor, type(pD)
\end{verbatim}
The sum of the monitor values gives the effective number of parameters
of the model, in the same way that the sum of the deviance monitor
values gives the total deviance of the model. 

Again, the values of the \verb+pD+ monitors are dumped to file using
\begin{verbatim}
monitors to "myfile.R", type(pD)
\end{verbatim}
When read into \R\ using the \verb+source()+ function, this creates
a list object named ``pD'' containing the sampled values.

\subsection{The \texttt{popt} monitor}

The \verb+popt+ monitor works exactly like the \verb+pD+ monitor, but
is created using the option \verb+type(popt)+.  It gives an estimate
of the optimism of the expected deviance ($p_{opt}$), which can be
added to the mean deviance to give the penalized expected deviance
\cite{plummer:2008}.

Under asymptotically favourable conditions in which $p_D \ll n$, where
$n$ is the sample size
\[
p_{opt} \approx 2 p_D
\]
For generalized linear models, a better approximation is
\[
p_{opt} \approx \sum_{i=1}^n \frac {p_{D_i}}{1 - p_{D_i}}
\]
where $p_D = \sum_i p_{D_i}$.

The \verb+popt+ monitor uses importance weights to estimate
$p_{opt}$. The resulting estimates may be numerically unstable when
$p_{D_i}$ is not small.  This typically occurs in random-effects
models, so it is recommended to check the output of the \verb+popt+
monitor with MCMC diagnostics.

\section{The msm module}

The \verb+msm+ module defines the matrix exponential function
\verb+mexp+ and the multi-state distribution \verb+dmstate+ which
describes the transitions between observed states in continuous-time
multi-state Markov transition models. 

\chapter{Functions}
\label{section:functions}

Functions allow deterministic nodes to be defined using the \verb+<-+
(``gets'') operator.  Most of the functions in \JAGS\ are scalar
functions taking scalar arguments. However, \JAGS\ also allows
arbitrary vector- and array-valued functions, such as the matrix
multiplication operator \verb+%*%+ and the transpose function
\verb+t()+ defined in the \verb+bugs+ module, and the matrix
exponential function \verb+mexp()+ defined in the \verb+msm+
module. \JAGS\ also uses an enriched dialect of the BUGS language with
a number of operators that are used in the S language.

Scalar functions taking scalar arguments are automatically vectorized.
They can also be called when the arguments are arrays with conforming
dimensions, or scalars. So, for example, the scalar $c$ can be added to
the matrix $A$ using
\begin{verbatim}
B <- A + c
\end{verbatim}
instead of the more verbose form
\begin{verbatim}
D <- dim(A)
for (i in 1:D[1])
   for (j in 1:D[2]) {
      B[i,j] <- A[i,j] + c
   }
}
\end{verbatim}

\section{Base functions}

The functions defined by the \verb+base+ module all appear as infix or
prefix operators. The syntax of these operators is built into the
\JAGS\ parser. They are therefore considered part of the modelling
language.  Table \ref{table:base:functions} lists them in reverse
order of precedence.

\begin{table}[h]
\begin{center}
\begin{tabular}{lll}
\hline
Type & Usage & Description\\ 
\hline
Logical           & \verb+x || y+ & Or \\
operators         & \verb+x && y+ & And \\
                  & \verb+!x+     & Not \\
\hline
Comparison  & \verb+x > y+ & Greater than\\
operators   & \verb+x >= y+ & Greater than or equal to  \\
            & \verb+x < y+ & Less than \\
            & \verb+x <= y+ & Less than or equal to \\
            & \verb+x == y+ & Equal \\
\hline
Arithmetic  & \verb-x + y- & Addition \\
operators   & \verb+x - y+ & Subtraction\\
            & \verb+x * y+ & Multiplication \\
            & \verb+x / y+ & Division \\
            & \verb+x %special% y+ &User-defined operators\\
            & \verb+-x+ & Unary minus\\
\hline
Power function & \verb+x^y+ & \\
\hline
\end{tabular}
\caption{Base functions listed in reverse order of precedence 
  \label{table:base:functions}}
\end{center}
\end{table}

Logical operators convert numerical arguments to logical values: zero
arguments are converted to FALSE and non-zero arguments to
TRUE. Logical and comparison operators return the value 1 if the
result is TRUE and 0 if the result is FALSE.  Comparison operators are
non-associative: the expression \verb+x < y < z+, for example, is
syntactically incorrect.

The \verb+%special%+ function is an exception in table
\ref{table:base:functions}. It is not a function defined by the
\verb+base+ module, but is a place-holder for any function
with a name starting and ending with the character ``\verb+%+'' Such
functions are automatically recognized as infix operators by the
\JAGS\ model parser, with precedence defined by table
\ref{table:base:functions}.

\section{Functions in the bugs module}

\subsection{Scalar functions}

Table \ref{table:bugs:scalar} lists the scalar-valued functions in the
\texttt{bugs} module that also have scalar arguments.  These functions
are automatically vectorized when they are given vector, matrix, or
array arguments with conforming dimensions.

Table \ref{table:bugs:link} lists the link functions in the
\texttt{bugs} module.  These are smooth scalar-valued functions that
may be specified using an S-style replacement function notation. So,
for example, the log link
\begin{verbatim}
log(y) <- x
\end{verbatim}
is equivalent to the more direct use of its inverse, the exponential
function:
\begin{verbatim}
y <- exp(x)
\end{verbatim}
This usage comes from the use of link functions in generalized linear
models.

\begin{table}
\begin{center}
\begin{tabular}{llll}
\hline
Usage  & Description & Value & Restrictions on arguments \\ 
\hline
\verb+abs(x)+       & Absolute value      & Real & \\
\verb+cos(x)+       & Cosine              & Real & \\
\verb+cloglog(x)+   & Complementary log log & Real & $0 < x < 1$ \\
\verb+equals(x,y)+  & Test for equality   & Logical & \\
\verb+exp(x)+       & Exponential         & Real & \\
\verb+icloglog(x)+  & Inverse complementary & Real & \\
                    & log log function    & \\
\verb+ilogit(x)+    & Inverse logit       & Real & \\
\verb+log(x)+       & Log function        & Real & $x > 0$ \\
\verb+logfact(x)+   & Log factorial       & Real & $x > -1$ \\
\verb+loggam(x)+    & Log gamma           & Real & $x > 0$ \\
\verb+logit(x)+     & Logit               & Real & $0 < x < 1$ \\
\verb+phi(x)+       & Standard normal cdf & Real & \\
\verb+pow(x,z)+     & Power function      & Real & If $x < 0$ then $z$ is integer \\ 
\verb+probit(x)+    & Probit              & Real & $0 < x < 1$ \\
\verb+round(x)+     & Round to integer    & Integer & \\
                    & away from zero      &      & \\
\verb+sin(x)+       & Sine                & Real & \\
\verb+sqrt(x)+      & Square-root         & Real & $x >= 0$ \\
\verb+step(x)+      & Test for $x \geq 0$ & Logical & \\
\verb+trunc(x)+     & Round to integer    & Integer & \\
                    & towards zero        & \\
\hline
\end{tabular}
\caption{Scalar functions in the \texttt{bugs} module \label{table:bugs:scalar}}
\end{center}
\end{table}

\begin{table}
\begin{center}
\begin{tabular}{llll}
\hline
Link function         & Description & Range & Inverse \\
\hline
\verb+cloglog(y) <- x+ & Complementary log log & $0 < y < 1$ & \verb+y <- icloglog(x)+ \\
\verb+log(y) <- x+    & Log           & $0 < y$ &  \verb+y <- exp(x)+ \\
\verb+logit(y) <- x+  & Logit         & $0 < y < 1$ &  \verb+y <- ilogit(x)+ \\
\verb+probit(y) <- x+ & Probit        & $0 < y < 1$ &  \verb+y <- phi(x)+\\
\hline
\end{tabular}
\caption{Link functions in the \texttt{bugs} module \label{table:bugs:link}}
\end{center}
\end{table}

\section{Scalar-valued functions with vector arguments}

Table \ref{table:bugs:scalar2} lists the scalar-valued functions in the
\texttt{bugs} module that take general arguments. Unless otherwise
stated in table \ref{table:bugs:scalar2}, the arguments to these functions
may be scalar, vector, or higher-dimensional arrays.

The \verb+max()+ and \verb+min()+ functions work like the
corresponding \R\ functions. They take a variable number of arguments
and return the maximum/minimum element over all supplied
arguments. This usage is  compatible with \WinBUGS, although more general.

\begin{table}
\begin{tabular}{lll}
\hline
Function & Description & Restrictions \\
\hline
\verb+inprod(x1,x2)+ & Inner product & Dimensions of $a$, $b$ conform \\
\verb+interp.lin(e,v1,v2)+ & Linear Interpolation & $e$ scalar, \\
                          &                     & $v1,v2$ conforming vectors \\
\verb+logdet(a)+ & Log determinant & $a$ is a square matrix \\
\verb+max(x1,x2,...)+ & Maximum element among all arguments & \\
\verb+mean(x)+  & Mean of elements of $a$ & \\
\verb+min(x1,x2,...)+ & Minimum element among all arguments & \\
\verb+prod(x)+  & Product of elements of $a$ & \\
\verb+sum(a)+   & Sum of elements of $a$& \\
\verb+sd(a)+    & Standard deviation of elements of $a$ & \\
\hline
\end{tabular}
\caption{Scalar-valued functions with general
  arguments in the \texttt{bugs} module \label{table:bugs:scalar2}}
\end{table}

\section{Vector- and array-valued functions}

Table \ref{table:bugs:vector} lists vector- or matrix-valued functions
in the \texttt{bugs} module.

The \texttt{sort} and \texttt{rank} functions behaves like their R
namesakes: \texttt{sort} accepts a vector and returns the same values
sorted in ascending order; \texttt{rank} returns a vector of ranks.
This is distinct from \WinBUGS, which has two scalar-valued functions
\verb+rank+ and \verb+ranked+.

\begin{table}
\begin{center}
\begin{tabular}{lll}
\hline
Usage & Description & Restrictions \\
\hline
\verb+inverse(a)+ & Matrix inverse & $a$ is a square matrix  \\
\verb+mexp(a)+ & Matrix exponential & $a$ is a square matrix \\
\verb+rank(v)+ & Ranks of elements of $v$ & $v$ is a vector   \\
\verb+sort(v)+ & Elements of $v$ in order & $v$ is a vector  \\
\verb+t(a)+    & Transpose                & $a$ is a matrix \\
\verb+a %*% b+  & Matrix multiplication & $a,b$ conforming vector or matrices\\

\hline
\end{tabular}
\caption{Vector- or matrix-valued functions in the \texttt{bugs}
  module \label{table:bugs:vector}}
\end{center}
\end{table}

\chapter{Distributions}
\label{section:distributions}

Distributions are used to define stochastic nodes using the \verb+~+
operator. The distributions defined in the bugs module are listed in
table \ref{table:bugs:distributions:real} (real-valued distributions),
\ref{table:bugs:distributions:discrete} (discrete-valued
distributions), and \ref{table:bugs:distributions:multi}
(multivariate distributions).

Some distributions have restrictions on the valid parameter values,
and these are indicated in the tables. If a Distribution object is
given invalid parameter values when evaluating the log-likelihood, it
returns $-\infty$. When a model is initialized, all stochastic nodes
are checked to ensure that the initial parameter values are valid for
their distribution.


\begin{table}
  \begin{center}
    \begin{tabular}{llcll}
      \hline
      Name & Usage & Density & Lower & Upper \\
      \hline
      Beta & \verb+dbeta(a,b)+ & 
      \multirow{2}{*}{
        $\frac{\textstyle x^{a-1}(1-x)^{b-1}}{\textstyle \beta(a,b)}$
      } & $0$ & $1$ \\
      & $a > 0$, $b > 0$ \\
      Chi-square & \verb+dchisqr(k)+ & 
      \multirow{2}{*}{
        $\frac{\textstyle x^{\frac{k}{2} - 1} \exp(-x/2)}
        {\textstyle 2^{\frac{k}{2}} \Gamma({\scriptstyle \frac{k}{2}})}$
      } & 0 & \\
      & $k > 0$ \\
      Double  & \verb+ddexp(mu,tau)+ & 
      \multirow{2}{*}{$\tau \exp(-\tau | x-\mu |)/2$} & & \\
      exponential & $\tau > 0$ \\
      Exponential & \verb+dexp(lambda)+ & 
      \multirow{2}{*}{$\lambda \exp(-\lambda x)$} & 0 & \\ 
      & $\lambda > 0$ \\
      Gamma       & \verb+dgamma(r, mu)+ & 
      \multirow{2}{*}{
        $\frac{\textstyle \mu^r x^{r - 1} \exp(-\mu x)}
        {\textstyle \Gamma(r)}$} & 0 & \\
      & $\mu > 0$, $r > 0$ \\
      Generalized & \verb+dgen.gamma(r,mu,beta)+ &  
      \multirow{2}{*}{
        $\beta \mu^{\beta r} x^{\beta r - 1}  \exp\{-(\mu x)^{\beta}\}$
      } & $0$ & \\
      gamma       & $\mu >0$, $\beta > 0$, $r > 0$ \\
      Log-normal  & \verb+dlnorm(mu,tau)+ & 
      \multirow{2}{*}{
        $\tau^{\frac{1}{2}} x^{-1} \exp \left\{-\tau (\log(x) - \mu)^2/2 \right\}$} & 0 \\
      ~ & $\tau > 0$ \\
      Normal   & \verb+dnorm(mu,tau)+ & 
      \multirow{2}{*}{
        $\left(\frac{\tau}{2\pi}\right)^{\frac{1}{2}} \exp\{-(x - \mu)^2 \tau\}$} & & \\
      ~ & $\tau > 0$ \\
      Pareto      & \verb+dpar(alpha, c)+ & 
      \multirow{2}{*}{
        $\alpha c^{\alpha} x^{-(\alpha + 1)}$
      } & $c$ & \\
      ~ & $\alpha > 0$, $c > 0$ \\
      Student t   & \verb+dt(mu,tau,k)+ & 
      \multirow{2}{*}{
        $\textstyle \frac{\Gamma(\frac{k+1}{2})}{\Gamma(\frac{k}{2})} 
        \left(\frac{\tau}{k\pi} \right)^{\frac{1}{2}} 
        \left\{1 + \frac{\tau (x - \mu)^2}{k} \right\}^{-\frac{(k+1)}{2}}$} & & \\
      ~ & $\tau > 0$, $k > 0$ \\
      Uniform     & \verb+dunif(a,b)+ & 
      \multirow{2}{*}{$\frac{\textstyle 1}{\textstyle b - a}$} & $a$ & $b$ \\
      ~ & $a < b$ \\ 
      Weibull     & \verb+dweib(v, lambda)+ & 
      \multirow{2}{*}{$v  \lambda  x^{v - 1} \exp (- \lambda x^v)$} & 0 & \\
      ~ & $v > 0$, $\lambda > 0$ \\
      \hline
    \end{tabular}
    \caption{Univariate real-valued distributions in the \texttt{bugs} module
      \label{table:bugs:distributions:real}}
  \end{center}
\end{table}

\begin{table}
  \begin{center}
    \begin{tabular}{llllll}
      \hline
      Name & Usage & Density & Lower & Upper \\
      \hline
      Bernoulli & \verb+dbern(p)+ & 
      \multirow{2}{*}{$p^x (1 - p)^{1 -x}$} & 
      $0$ & $1$ \\
      ~ & $0 < p < 1$ \\
      Binomial  & \verb+dbin(p,n)+ & 
      %\multirow{2}{*}{$\begin{array}{c} n \\ x \end{array} p^x (1-p)^{n-x}$}
      \multirow{2}{*}{${n \choose x}  p^x (1-p)^{n-x}$}
      ~  & $0$ & $n$ \\
      ~ & $0 < p < 1$, $n \in \mathbb{N}^*$ \\
      Categorical & \verb+dcat(p)+ & \multirow{2}{*}{$\frac{\textstyle p_x}{\textstyle \sum_i p_i}$} & $1$ & $N$ \\
      ~ & $p \in (\mathbb{R}^+)^N$  \\
      Hypergeometric & \verb+dhyper(n1,n2,m1,psi)+ &
      \multirow{2}{*}{
        $\textstyle {n_1 \choose x} {n_2 \choose m_1 - x} \psi^x$
      } &
      $\text{max}(0,n_+ - m_1)$ & $\text{min}(n_1,m_1)$ \\
      ~ & $0 \leq n_i$, $0 < m_1 \leq n_+$  \\
      Negative & \verb+dnegbin(p, r)+ &
      \multirow{2}{*}{${x + r -1 \choose x} p^r (1-p)^x$} & 0 & \\
      binomial & $0 < p < 1$, $r \in \mathbb{N}^+$ \\
      Poisson & \verb+dpois(lambda)+ & 
      \multirow{2}{*}{$\frac{\textstyle \exp(-\lambda) \lambda^x}{\textstyle x!}$} & 0 & \\
      ~ & $\lambda > 0$ \\
      \hline
    \end{tabular}
  \caption{Discrete univariate distributions in the \texttt{bugs} module
    \label{table:bugs:distributions:discrete}}
  \end{center}
\end{table}


\begin{table}
  \begin{center}
    \begin{tabular}{lll}
      \hline
      Name & Usage & Density \\
      \hline
      Dirichlet & \verb+p ~ ddirch(alpha)+ & 
      \multirow{2}{*}{$\Gamma(\sum_i \alpha_i) \prod_j 
        \frac{\textstyle p_j^{\alpha_j - 1}}{\textstyle \Gamma(\alpha_j)}$} \\
      ~ & $\alpha_j \geq 0$ \\
      & \\
      Multivariate & \verb+x ~ dmnorm(mu,Omega)+ &
      \multirow{2}{*}{
        $\left(\frac{|\Omega|}{2\pi}\right)^{\frac{1}{2}} exp\{-(x-\mu)^T \Omega (x-\mu) / 2\}$} \\
      normal & $\Omega$ positive definite \\
      Wishart & \verb+Omega ~ dwish(R,k)+ &
      \multirow{2}{*}{
        $\frac{\textstyle |\Omega|^{(k-p-1)/2} |R|^{k/2} \exp\{-\text{Tr}(R\Omega/2)\}}
               {\textstyle 2^{pk/2} \Gamma_p (k/2)}$
      } \\
      & $R$ pos. def. \\
      Multivariate & \verb+x ~ dmt(mu,Omega,k)+ &
      \multirow{2}{*}{
        $\frac{\textstyle \Gamma \{(k+p)/2\}}{\textstyle \Gamma(k/2) (n\pi)^{p/2}}
        |\Omega|^{1/2}
        \left\{1 + \frac{1}{k} (x - \mu)^T \Omega (x - \mu) \right\}^{-\frac{(k+p)}{2}}$   } \\
      Student t &  $\Omega$ pos. def. & \\
      Multinomial  & \verb+x ~ dmulti(p, n)+ & 
      \multirow{2}{*}{$n! \prod_j 
        \frac{\textstyle p_j^{x_j}}{\textstyle x_j!}$} \\
      ~ & $\sum_i x_i = n$ \\
      & \\
    \hline
    \end{tabular}
    \caption{Multivariate distributions in the \texttt{bugs} module
      \label{table:bugs:distributions:multi}}
  \end{center}
\end{table}

\chapter{Differences between \JAGS\ and \WinBUGS}

Although \JAGS\ aims for the same functionality as \WinBUGS, there are
a number of important differences.

\subsection{Data format}

There is no need to transpose matrices and arrays when transferring
data between \R\ and \JAGS, since \JAGS\ stores the values of an array
in ``column major'' order, like \R\ and FORTRAN ({\em i.e.} filling
the left-hand index first).

If you have an \textsf{S}-style data file for \WinBUGS\ and you wish
to convert it for \JAGS, then use the command \texttt{bugs2jags},
which is supplied with the \CODA\ package.


\subsection{Samplers}

\JAGS\ has a more limited set of samplers than \WinBUGS, which leads
to relatively poor performance. 

\subsection{Distributions}

Structural zeros are allowed in the Dirichlet distribution. If
\begin{verbatim}
p ~ ddirch(alpha)
\end{verbatim}
and some of the elements of alpha are zero, then the corresponding
elements of p will be fixed to zero.

The Multinomial (\verb+dmulti+) and Categorical (\verb+dcat+)
distributions, which take a vector of probabilities as a parameter,
may use unnormalized probabilities. The probability vector is
normalized internally so that
\[
p_i \rightarrow \frac{p_i}{\sum_j p_j}
\]

\subsection{Observable Functions}
\label{section:obfun}

Logical nodes in the \BUGS\ language are a convenient way of
describing the relationships between observables (constant and
stochastic nodes), but are not themselves observable. You cannot
supply data values for a logical node.  

This restriction can occasionally be inconvenient, as there are
important cases where the data are a deterministic function of
unobserved variables.  Two important examples are
\begin{enumerate}
\item Censored data, which commonly occurs in survival analysis. In
the most general case, we know that unobserved failure time $T$
lies in the interval $(L,U]$.
\item Aggregate data when we observe the sum of two or more
unobserved variables.
\end{enumerate}
\JAGS\ contains two novel distributions to handle these situations.  
\begin{enumerate}
\item The \texttt{dinterval} distribution represents interval-censored
data. It has two parameters: $t$ the original continuous variable, and
$c[]$, a vector of cut points of length $M$, say. If \texttt{X $\sim$
dinterval(t, c)} then 

\begin{tabular}{lll}
$X=0$   & if & $t < c[1]$\\
$X=m$   & if & $c[m] \leq t < c[m+1]$ for $1 \leq m < M$\\
$X = M$ & if & $c[M] \leq t$.
\end{tabular}

\item The \texttt{dsum} distribution represents the sum of two
variables.  It has two parameters, $x1$ and $x2$. If \texttt{Y $\sim$
dsum(x1,x2)} then $Y=x1+x2$.
\end{enumerate}
These distributions exist to give a likelihood to data that is, in fact,
a deterministic function of the parameters.  The relation
\begin{verbatim}
Y ~ dsum(x1, x2)
\end{verbatim}
is logically equivalent to
\begin{verbatim}
Y <- x1 + x2
\end{verbatim}
But the latter form does not create a contribution to the likelihood,
and does not allow you to define $Y$ as data.  The likelihood function
is trivial: it is 1 if the parameters are consistent with the data and
0 otherwise.  The \texttt{dsum} distribution also requires a special
sampler, which can currently only handle the case where both $x1$ and
$x2$ are discrete-valued.

\subsection{Data transformations}
\label{section:data:tranformations}

\JAGS\ allows data transformations, but the syntax is different from
\BUGS.  \BUGS\ allows you to put a stochastic node twice on the left
hand side of a relation, as in this example taken from the manual
\begin{verbatim}
   for (i in 1:N) {
      z[i] <- sqrt(y[i])
      z[i] ~ dnorm(mu, tau)
   }
\end{verbatim}
This is forbidden in \JAGS. You must put data transformations in a 
separate block of relations preceded by the keyword \texttt{data}:
\begin{verbatim}
data {
   for (i in 1:N) {
      z[i] <- sqrt(y[i])
   }
}
model {
   for (i in 1:N) {
      z[i] ~ dnorm(mu, tau)
   }
   ...
}
\end{verbatim}
This syntax preserves the declarative nature of the \BUGS\ language.
In effect, the data block defines a distinct model, which describes
how the data is generated. Each node in this model is forward-sampled
once, and then the node values are read back into the data table. The
data block is not limited to logical relations, but may also include
stochastic relations. You may therefore use it in simulations,
generating data from a stochastic model that is different from the one
used to analyse the data in the \texttt{model} statement.

This example shows a simple location-scale problem in which the ``true''
values of the parameters \texttt{mu} and \texttt{tau} are generated
from a given prior in the \texttt{data} block, and the generated
data is analyzed in the \texttt{model} block.
\begin{verbatim}
data {
   for (i in 1:N) {
      y[i] ~ dnorm(mu.true, tau.true) 
   }
   mu.true ~ dnorm(0,1);
   tau.true ~ dgamma(1,3);
}
model {
   for (i in 1:N) {
      y[i] ~ dnorm(mu, tau)
   }
   mu ~ dnorm(0, 1.0E-3)
   tau ~ dgamma(1.0E-3, 1.0E-3)
}
\end{verbatim}
Beware, however, that every node in the \texttt{data} statement will
be considered as data in the subsequent \texttt{model} statement. This
example, although superficially similar, has a quite different interpretation.
\begin{verbatim}
data {
   for (i in 1:N) {
      y[i] ~ dnorm(mu, tau) 
   }
   mu ~ dnorm(0,1);
   tau ~ dgamma(1,3);
}
model {
   for (i in 1:N) {
      y[i] ~ dnorm(mu, tau)
   }
   mu ~ dnorm(0, 1.0E-3)
   tau ~ dgamma(1.0E-3, 1.0E-3)
}
\end{verbatim}
Since the names \texttt{mu} and \texttt{tau} are used in both
\texttt{data} and \texttt{model} blocks, these nodes will be
considered as {\em observed} in the model and their values will be
fixed at those values generated in the \texttt{data} block.

\subsection{Directed cycles}

Directed cycles are forbidden in \JAGS. There are two important
instances where directed cycles are used in \BUGS.
\begin{itemize}
\item Defining autoregressive priors
\item Defining ordered priors
\end{itemize}
For the first case, the \texttt{GeoBUGS} extension to \WinBUGS\ provides
some convenient ways of defining autoregressive priors. These should be
available in a future version of \JAGS.

\subsection{Censoring, truncation and prior ordering}
\label{section:censoring}

These are three, closely related issues that are all handled using
the \texttt{I(,)} construct in \BUGS.

Censoring occurs when a variable $X$ is not observed directly,
but is observed only to lie in the range $(L,U]$.  Censoring is
an {\em a posteriori} restriction of the data, and is represented
in WinBUGS by the \texttt{I(,)} construct, {\em e.g.}
\begin{verbatim}
X ~ dnorm(theta, tau) I(L,U)
\end{verbatim}
where $L$ and $U$ are constant nodes.

Truncation occurs when a variable is known {\em a priori} to lie in
a certain range.  Although \BUGS\ has no construct for representing
truncated variables, it turns out that there is no difference between
censoring and truncation for top-level parameters ({\em i.e.} variables
with no unobserved parents).  Hence, for example, this
\begin{verbatim}
theta ~ dnorm(0, 1.0E-3) I(0, )
\end{verbatim}
is a perfectly valid way to describe a parameter $\theta$ with a
half-normal prior distribution.

Prior ordering occurs when a vector of nodes is known {\em a priori}
to be strictly increasing or decreasing. It can be represented in
WinBUGS with symmetric $I(,)$ constructs,  {\em e.g.}
\begin{verbatim}
X[1] ~ dnorm(0, 1.0E-3) I(,X[2])
X[2] ~ dnorm(0, 1.0E-3) I(X[1],)
\end{verbatim}
ensures that $X[1] \leq X[2]$.

\JAGS\ makes an attempt to separate these three concepts.

Censoring is handled in \JAGS\ using the new distribution
\texttt{dinterval} (section \ref{section:obfun}). This can be
illustrated with a survival analysis example.  A right-censored
survival time $t_i$ with a Weibull distribution is described in
\WinBUGS\ as follows:
\begin{verbatim}
t[i] ~ dweib(r, mu[i]) I(c[i], )
\end{verbatim}
where $t_i$ is unobserved if $t_i > c_i$.  In \JAGS\ this becomes
\begin{verbatim}
is.censored[i] ~ dinterval(t[i], c[i])
t[i] ~ dweib(r, mu[i])
\end{verbatim}
where \verb+is.censored[i]+ is an indicator variable that takes the
value 1 if $t_i$ is censored and 0 otherwise. See the MICE and KIDNEY
examples in the ``classic bugs'' set of examples.

Truncation is represented in \JAGS\ using the \texttt{T(,)} construct,
which has the same syntax as the \texttt{I(,)} construct in \WinBUGS,
but has a different interpretation. If
\begin{verbatim}
X ~ dfoo(theta) T(L,U)
\end{verbatim}
then {\em a priori} $X$ is known to lie between $L$ and $U$. This
generates a likelihood
\[
\frac{p(x \mid \theta)}{P(L \leq X \leq U \mid \theta)}
\]
if $L \leq X \leq U$ and zero otherwise, where $p(x \mid \theta)$ is
the density of $X$ given $\theta$ according to the distribution
\texttt{foo}. Note that calculation of the denominator may be
computationally expensive.

Prior ordering of top-level parameters in the model can be achieved
using the \texttt{sort} function, which sorts a vector in ascending
order.

Symmetric truncation relations like this
\begin{verbatim}
alpha[1] ~ dnorm(0, 1.0E-3) I(,alpha[2])
alpha[2] ~ dnorm(0, 1.0E-3) I(alpha[1],alpha[3])
alpha[3] ~ dnorm(0, 1.0E-3) I(alpha[2],)
\end{verbatim}
Should be replaced by this
\begin{verbatim}
for (i in 1:3) {
   alpha0[i] ~ dnorm(0, 1.0E-3)
}
alpha[1:3] <- sort(alpha0)
\end{verbatim}

\chapter{Development}

At some point there will be a separate manual for \JAGS\ developers.
At the moment, if you want to start hacking \JAGS\ then you must rely
on the source file documentation.  This is written in JavaDoc style
and can be processed using \texttt{doxygen} to generate a reference
manual in the \texttt{doc} directory.

The \JAGS\ source is divided into three main directories:
\texttt{lib}, \texttt{modules}, and \texttt{terminal}. The
\texttt{lib} directory contains the \JAGS\ library, which contains all
the facilities for defining a Bayesian graphical model in the \BUGS\
language, running the Gibbs sampler and monitoring the sampled
values. The \JAGS\ library is divided into several convenience
libraries
\begin{description}
\item[sarray] which defines the basic SArray class, modelled on an
\textsf{S} language array, and its associated classes.
\item[function] which defines the interface for functions and
the \texttt{FuncTab} class that allows you to reference them by name.
\item[distribution] which defines the interface for distribution and
the \texttt{DistTab} class that allows you to reference them by name.
\item[graph] which defines the various Node classes used by \JAGS\
when constructing a Bayesian graphical model, as well as the \texttt{Graph}
class which is a container for nodes.
\item[sampler] which defines the interface for Samplers, which update
stochastic nodes in the graph.
\item[model] which defines all the classes needed to create a model,
including monitor classes.
\item[compiler] which contains the Compiler class and a number of 
supporting classes designed for an efficient translation of a BUGS-language
description the model into a \texttt{Graph}.
\item[rng] which defines the interface for random number generators (RNGs)
and the factories that create them.
\item[util] which contains some utility functions used in the rest of
the \JAGS\ library.
\end{description}
The \texttt{Console} class provides a clean interface to the \JAGS\
library.  The member functions of the \texttt{Console} class conduct
all of the operations one may wish to do on a Bayesian graphical
model.  They are designed to catch any exceptions thrown by the
library and print an informative message to either an output stream or
an error stream, depending on the result.

The \texttt{modules} directory contains the source code for JAGS
modules, which contain concrete classes corresponding to the abstract
classes defined in the \JAGS\ library.

The \texttt{terminal} directory contains the source code for a
reference front end for the \JAGS\ library, which uses the
\textsf{Stata}-like syntax described in section
\ref{section:scripting}.

\chapter{Feedback}

Please send feedback to \url{martyn_plummer@users.sourceforge.net}.
I am particularly interested in the following problems:

\begin{itemize}
\item Crashes, including both segmentation faults and uncaught exceptions.
\item Incomprehensible error messages
\item Models that should compile, but don't 
\item Output that cannot be validated against \WinBUGS
\item Documentation erors
\end{itemize}

If you want to send a bug report, it must be reproducible. Send the
model file, the data file, the initial value file and a script file
that will reproduce the problem. Describe what you think should
happen, and what did happen.

\chapter{Acknowledgments}

Many thanks to the \BUGS\ development team, without whom \JAGS\ would
not exist.  Thanks also to Simon Frost for pioneering \JAGS\ on
Windows and Bill Northcott for getting \JAGS\ on Mac OS X to
work. Kostas Oikonomou found many bugs while getting \JAGS\ to work on
Solaris using Sun development tools and libraries.  Bettina Gruen,
Chris Jackson, Greg Ridgeway and Geoff Evans also provided useful
feedback.  Special thanks to Jean-Baptiste Denis who has been very
diligent in providing feedback on JAGS and who kindly proof-read this
manual.

Testing of \JAGS\ on IRIX 6.5 was carried out on Helix Systems at the
National Institutes of Health, Bethesda, MD (\url{http://helix.nih.gov}).

\bibliographystyle{plain}
\bibliography{jags_user_manual}

\clearpage

\appendix
\chapter{Installation}

\JAGS\ is distributed in binary format for Microsoft Windows, Mac OS
X, and most Linux distributions.  The following instructions are for
those who wish to build \JAGS\ from source.

\JAGS\ has been successfully built on GNU/Linux, FreeBSD, Windows XP,
Mac OS X, IRIX and Solaris. If you manage to build \JAGS\ on any other
platform, then please let me know at
\url{martyn_plummer@users.sourceforge.net}.

These instructions assume that you in a unix environment. If you
are working on Windows, then you should skip straight to section
\ref{section:windows}

\section{Configuration}

\JAGS\ follows the usual GNU convention of 
\begin{verbatim}
./configure
make
make install
\end{verbatim}
which is described in more detail in the file \texttt{INSTALL} in
the top-level source directory. On some UNIX platforms, you may
be required to use GNU make (gmake) instead of the native make
command. On systems with multiple processors, you may use the option 
\verb+-j+ to speed up compilaition, {\em e.g.} for a quad-core PC you
may use:
\begin{verbatim}
make -j4
\end{verbatim}

At configure time you also have the option of defining options such
as:
\begin{itemize}
\item The names of the C, C++, and Fortran compilers.  Although
  \JAGS\ contains no Fortran code, you are required to define a
  Fortran compiler so that \JAGS\ modules can be linked against
  libraries written in Fortran (such as BLAS and LAPACK)
\item Optimization flags for the C and C++ compilers.  \JAGS\ is
  optimized by default if the GNU compiler (gcc) is used. Otherwise
  you must explicitly supply optimization flags.
\item Installation directories. \JAGS\ conforms to the GNU standards
  for where files are installed. You can control the installation
  directories in more detail using the flags that are listed when
  you type \verb+./configure --help+.
\end{itemize}

\subsection{Configuration for a private installation}

If you do not have administrative privileges, you may wish to install
\JAGS\ in your home directory. This can be done with the following
configuration options
\begin{verbatim}
export JAGS_HOME=$HOME/jags #or wherever you want it
./configure --bindir=$JAGS_HOME/bin --libdir=$JAGS_HOME/lib \
 --libexecdir=$JAGS_HOME/bin --includedir=$JAGS_HOME/include
\end{verbatim}
You then need to modify your PATH environment variable to include
\verb+$JAGS_HOME/bin+. You may also need to set \verb+LD_LIBRARY_PATH+
to include \verb+$JAGS_HOME/lib+ (On Linux this is not necessary as
the location of libjags and libjrmath is hard-coded into the
\JAGS\ binary).

\section{BLAS and LAPACK}
\label{section:blas:lapack}

BLAS (Basic Linear Algebra System) and LAPACK (Linear Algebra Pack)
are two libraries of routines for linear algebra. They are used by the
multivariate functions and distributions in the \texttt{bugs} module.
Most unix-like operating system vendors supply shared libraries that
provide the BLAS and LAPACK functions, although the libraries may not
literally be called ``blas'' and ``lapack''.  During configuration, a
default list of these libraries will be checked. If \texttt{configure}
cannot find a suitable library, it will stop with an error message.

You may use alternative BLAS and LAPACK libraries using the configure
options \texttt{--with-blas} and \texttt{--with-lapack}
\begin{verbatim}
./configure --with-blas="-lmyblas" --with-lapack="-lmylapack"
\end{verbatim}

If the BLAS and LAPACK libraries are in a directory that is not on the
default linker path, you must set the \verb+LDFLAGS+ environment variable
to point to this directory at configure time:
\begin{verbatim}
LDFLAGS="-L/path/to/my/libs" ./configure ...
\end{verbatim}
If your BLAS and LAPACK libraries depend on other libraries that are
not on the linker path, you must supply these dependency libraries as
additional arguments to \texttt{--with-blas} and
\texttt{--with-lapack} (See section \ref{section:gnulinux} for some
examples of this).

At runtime, if you have linked \JAGS\ against BLAS or LAPACK in
a non-standard location, you must supply this location with the
environment variable \verb+LD_LIBRARY_PATH+, {\em e.g.}
\begin{verbatim}
LD_LIBRARY_PATH="/path/to/my/libs:${LD_LIBRARY_PATH}"
\end{verbatim} %$
Alternatively, you may hard-code the paths to the blas and lapack
libraries at compile time. This is compiler and platform-specific,
but is typically achieved with
\begin{verbatim}
LDFLAGS="-L/path/to/my/libs -R/path/to/my/libs
\end{verbatim}

Note: You should not link to BLAS and LAPACK libraries by giving
the shared object files as arguments to \texttt{--with-blas} and
\texttt{--with-lapack}:
\begin{verbatim}
#DON'T DO THIS!
./configure --with-blas=/path/to/my/libblas.so \
--with-lapack=/path/to/my/liblapack.so 
\end{verbatim}
Although \JAGS\ will compile correctly, the resulting \texttt{bugs}
module will not be linked to your BLAS or LAPACK libraries.  Although
this can be rectified using the \verb+LD_PRELOAD+ environment variable
it is not recommended.

\section{Platform-specific notes}

\subsection{GNU/Linux}
\label{section:gnulinux}

\subsubsection{Fortran compiler}

The GNU FORTRAN compiler changed between gcc 3.x and gcc 4.x from
\verb+g77+ to \verb+gfortran+. Code produced by the two compilers is
binary incompatible. If your BLAS and LAPACK libraries are linked
against \verb+libgfortran+, then they were built with \verb+gfortran+
and you must also use this to compile \JAGS. 

Most recent GNU/Linux distributions have moved completely to gcc 4.x.
However, some older systems may have both compilers installed.
Unfortunately, if \verb+g77+ is on your path then the configure script
will find it first, and will attempt to use it to build \JAGS. This
results in a failure to recognize the installed BLAS and LAPACK
libraries. In this event, set the \verb+F77+ variable at configure time.
\begin{verbatim}
F77=gfortran ./configure
\end{verbatim}

\subsubsection{BLAS and LAPACK}

The {\bf BLAS} and {\bf LAPACK} libraries from Netlib
(\url{http://www.netlib.org}) should be provided as part of your Linux
distribution. If your Linux distribution splits packages into ``user''
and ``developer'' versions, then you must install the developer
package ({\em e.g.}  \texttt{blas-devel} and \texttt{lapack-devel}).

{\bf Suse Linux Enterprise Server (SLES)} does not include BLAS and
LAPACK in the main distribution. They are included in the SLES SDK, on
a set of CD/DVD images which can be downloaded from the Novell web
site.  See \url{http://developer.novell.com/wiki/index.php/SLES_SDK}
for more information.

It is quite common for the Netlib implementations of BLAS and LAPACK
to break when they are compiled with the latest GNU compilers.  Linux
distributions that use ``bleeding edge'' development tools -- such as
{\bf Fedora} -- may ship with a broken version of BLAS and
LAPACK. Normally, this problem is quickly identified and
fixed. However, you need to take care to use the online updates of the
BLAS and LAPACK packages from your Linux Distributor, and not rely on
the version that came on the installation disk.

\subsubsection{ATLAS}

On Fedora Linux, pre-compiled atlas libraries are available via the
\texttt{atlas} and \texttt{atlas-devel} RPMs.  These RPMs install the
atlas libraries in the non-standard directory \texttt{/usr/lib/atlas}
(or \texttt{/usr/lib64/atlas} for 64-bit builds) to avoid conflicts
with the standard \texttt{blas} and \texttt{lapack} RPMs. To use the
atlas libraries, you must supply their location using the
\verb+LDFLAGS+ variable (see section \ref{section:blas:lapack})
\begin{verbatim}
./configure LDFLAGS="-L/usr/lib/atlas"
\end{verbatim}
Runtime linking to the correct libraries is ensured by the automatic
addition of \texttt{/usr/lib/atlas} to the linker path (see the file
\texttt{/etc/ld.so.conf}), so you do not need to set the
environment variable \verb+LD_LIBRARY_PATH+ at run time.

\subsubsection{AMD Core Math Library}
\label{section:acml:linux}

The AMD Core Math Library (acml) provides optimized BLAS and LAPACK
routines for AMD processors. To link \JAGS\ with \texttt{acml}, you
must supply the \texttt{acml} library, {\em and its dependencies}, as
arguments to \texttt{--with-blas}.  It is not necessary to set the
\texttt{--with-lapack} argument as \texttt{acml} provides both sets of
functions. See also section~\ref{section:blas:lapack} for run-time
instructions.

For example, to link to the 64-bit acml using gcc 4.0+:
\begin{verbatim}
LDFLAGS="-L/opt/acml4.3.0/gfortran64/lib" \
./configure --with-blas="-lacml -lacml_mv -lgfortran" 
\end{verbatim}
The library \verb+acmv_mv+ library is a vectorized math library that
exists only for the 64-bit version and is omitted when linking against
32-bit acml.

On multi-core systems, you may wish to use the threaded acml library.
To do this, link to \verb+acml_mp+ and add the compiler flag
\verb+-fopenmp+:
\begin{verbatim}
LDFLAGS="-L/opt/acml4.3.0/gfortran64_mp/lib" \
CXXFLAGS="-O2 -g -fopenmp" ./configure --with-blas="-lacml_mp -lacml_mv -lgfortran" 
\end{verbatim}
The number of threads used by multi-threaded acml may be controlled
with the environment variable \verb+OMP_NUM_THREADS+.

For older Linux systems, the last version that supports gcc 3.4 is
\texttt{acml} 3.6.0. When using gcc 3.4, link against \texttt{libg2c}.
\begin{verbatim}
LDFLAGS="-L/opt/acml3.6.0/gnu64/lib" \
./configure --with-blas="-lacml -lacml_mv -lg2c"
\end{verbatim}

\subsubsection{Intel Math Kernel Library}

The Intel Math Kernel library (MKL) provides optimized BLAS and LAPACK
routines for Intel processors. The instructions below are for MKL
version 10.0 and above which use a ``pure layered'' model for linking.
The layered model gives the user fine-grained control over four
different library layers: interface, threading, computation, and
run-time library support. Some examples of linking to MKL using this
layered model are given below. These examples are for GCC compilers on
\verb+x86_64+. The choice of interface layer is important on
\verb+x86_64+ since the Intel Fortran compiler returns complex values
differently from the GNU Fortran compiler. You must therefore use the
interface layer that matches your compiler (\verb+mkl_intel*+ or \verb+mkl_gf*+).  
The use of MKL is incompatible with building a shared library for
\JAGS. It can be used if \JAGS\ is configured with the option
\verb+--disable-shared+., This builds an executable that is statically
linked to the \JAGS\ library and modules.

JAGS can be linked to a sequential version of MKL by
\begin{verbatim}
MKL_HOME=/opt/intel/mkl/10.0.3.020/
MKL_LIB_PATH=${MKL_HOME}/lib/em64t/
./configure --disable-shared \
            --with-blas="-L${MKL_LIB_PATH} -lmkl_gf_lp64 -lmkl_sequential -lmkl_core"
\end{verbatim}

Threaded MKL may be used with:
\begin{verbatim}
./configure --disable-shared\
   --with-blas="-L${MKL_LIB_PATH} -lmkl_gf_lp64 -lmkl_gnu_thread -lmkl_core -liomp5 -lpthread"
\end{verbatim}
The default number of threads will be chosen by the OpenMP software,
but can be controlled by setting \verb+OMP_NUM_THREADS+ or \verb+MKL_NUM_THREADS+.

\subsubsection{Using Intel Compilers}

\JAGS\ has been successfully built with the Intel C, C++ and Fortran
compilers.  The additional configure options required to use the Intel
compilers are:
\begin{verbatim}
source /opt/intel/Compiler/11.1/bin/ifortvars.sh
source /opt/intel/Compiler/11.1/bin/iccvars.sh
CC=icc CXX=icpc F77=ifort ./configure 
\end{verbatim}

\subsection{OpenSolaris}

\JAGS\ has been successfully built and tested on the Intel x86
platform under OpenSolaris 2008.05 using the Sun Studio Express 5/08
compilers.
\begin{verbatim}
./configure CC=cc CXX=CC F77=f95 \
CFLAGS="-xO3 -xarch=sse2" \
FFLAGS="-xO3 -xarch=sse2" \
CXXFLAGS="-xO3 -xarch=sse2"
\end{verbatim}
The Sun Studio compiler is not optimized by default. Use the option
\verb+-xO3+ for optimization (NB This is the letter ``O'' not the
number ``0'') In order to use the optimization flag \verb+-xO3+ you
must specify the architecture with the \verb+-xarch+ flag. The options
above are for an Intel processor with SSE2 instructions. This must be
adapted to your own platform.

To compile a 64-bit version of JAGS, add the option \verb+-m64+ to
all the compiler flags.

Solaris provides two versions of the C++ standard library:
\texttt{libCstd}, which is the default, and \texttt{libstlport4},
which conforms more closely to the C++ standard. \JAGS\ may be linked
to the stlport4 library by adding the options
\verb+-library=stlport4 -lCrun+ to \verb+CXXFLAGS+.

The configure script automatically detects the Sun Performance library,
which implements the BLAS/LAPACK functions.  Automatic detection may
not work on older versions of Sun Studio, which used a different syntax
for specifying this library.  In this case, you may need to use the
configure option
\begin{verbatim}
--with-blas="-xlic_lib=sunperf -lsunmath"
\end{verbatim}

\subsubsection{Using acml}

AMD provides a version of their Core Math Library (acml) for
Solaris. To use this library instead of the Sun Performance library
add the following configure options (changing paths as appropriate):
\begin{verbatim}
--with-blas="-lacml -lacml_mv -lfsu" \
LDFLAGS="-L/opt/acml4.1.0/sun64/lib \
-R/opt/acml4.1.0/sun64/lib:/opt/SunStudioExpress/lib"
\end{verbatim}
The acml library is only available in 64-bit mode, so the option
\verb+-m64+ must also be added to all the compiler flags.

As with using acml on Linux (section \ref{section:acml:linux}), the
configure option \verb+--with-blas+ must include not only the acml
library, but also its dependencies. The \verb+LDFLAGS+ option \verb+-R+
hard-codes the paths to these libraries into the JAGS modules that
require them.

\subsection{IRIX}


\JAGS\ has been successfully built using the MIPSpro 7.4 compiler on
IRIX 6.5. The following configure options were used:
\begin{verbatim}
./configure CC=cc CXX=CC F77=f77 \
CFLAGS="-O2 -g2 -OPT:IEEE_NaN_inf=ON" \
CXXFLAGS="-O2 -g2 -OPT:IEEE_NaN_inf=ON" 
\end{verbatim}
and \JAGS\ was built with \verb+gmake+ (GNU make).

BLAS and LAPACK functions on IRIX are provided by the Scientific
Computing Software library (\verb+scs+). The presence of this library
is detected automatically by the configure script.

When using the MIPSpro compiler, optimization flags must be given
explicitly at configure time. If this is not done, then \JAGS\ will
not be optimized at all and will run slowly.

\subsection{Mac OS X: instructions from Bill Northcott}

If trying to build software on Mac OS X you really need to use Leopard
(10.5.x) or Snow Leopard (10.6.x). Unless otherwise stated these
instruction assume Snow Leopard (10.6.x). The open source support has
improved greatly in recent releases. You also need the latest version
of Apple's Xcode development tools. The current version is Xcode 3.2.x
(Leopard uses 3.1.x).  Early versions have serious bugs which affect R
and \JAGS.  Xcode is available as a free download from
\url{http://developer.apple.com}. You need to set up a free login to
ADC. The Apple developer tools do not include a Fortran
compiler. Without Fortran, you will not be able to build \JAGS.

For instructions for building on Tiger or for older versions of
\R\ see previous versions of this manual.

The GNU gfortran Fortran compiler is included in the \R\ binary
distribution available on CRAN. Install the \R\ binary and select all
the optional components in the `Customize' step of the installer.
These instructions assume R-2.7.x.

The default C/C++ compiler for Snow Leopard is gcc-4.2.x. Xcode 3.2
also includes gcc-4.2 and llvm-gcc4.2.  The code has been successfully
built with these optional compilers but will only run on Leopard.
llvm is being actively developed by Apple and may produce better code.

MacOS X 10.2 and onwards include optimised versions of the BLAS and
LAPACK libraries.  So no extra libraries are is needed for Snow
Leopard.  Optimisation continues andApple are working on using GPUs
for this sort of math.  Make sure your OS is up to date.

To ensure the \JAGS\ configure script can find the Fortran compiler
for a bash shell
\begin{verbatim}
export F77=/usr/local/bin/gfortran
\end{verbatim}

On 64 bit hardware, which means most recent Macs, there may be a 
problem with the Fortran compiler.  Apple's compilers default to 64 bit 
builds on 64 bit hardware but the Fortran binaries available default to 
32 bit builds. This means you need to add compile and link options.

For instance on 64 bit Intel Macs type
\begin{verbatim}
export CFLAGS='-arch x86_64'
export CXXFLAGS='-arch x86_64'
export FFLAGS='-arch x86_64'
export LDFLAGS='-arch x86_64'
\end{verbatim}

Some Fortran compilers (not the ones from CRAN) do not understand 
the -arch option. For these you will need something like:
\begin{verbatim}
export CFLAGS='-arch x86_64'
export CXXFLAGS='-arch x86_64'
export FFLAGS='-m64'
export LDFLAGS='-arch x86_64'
\end{verbatim}

To build \JAGS\ unpack the source code and cd into the source  
directory. Type the following:
\begin{verbatim}
./configure
make
\end{verbatim}
(if you have multiple CPUs try \verb+make -j 4+ or
\verb+make -j 8+. It may need to be issued more than once)
\begin{verbatim}
sudo make install
\end{verbatim}

You need to ensure \texttt{/usr/local/bin} is in your PATH in order
for `jags' to work from a shell prompt.

This will build the default architecture for you Mac: ppc on a G4 or
G5 and i386 or \verb+x86_64+ on an Intel Mac.  If you want to build
multiple architecture fat binaries, you will need to ensure that
libtool in the JAGS sources is version 1.5.24 or later.  Then you can
use configure commands like
\begin{verbatim}
CXXFLAGS="-arch i386 -arch x86_64" ./configure
\end{verbatim}

Make will then build fat binaries.  See the R Mac developers page 
\url{http://r.research.att.com/} for instructions to build fat R packages.

A final note on MacOS X builds: do NOT use \texttt{-O3}.  It is not optimal 
and may find compiler bugs.  Apple recommends \texttt{-Os}.

\subsection{Windows}
\label{section:windows}

These instructions use MinGW, The Minimalist GNU system for Windows.
You need some familiarity with Unix in order to follow the build
instructions but, once built, \JAGS\ can be installed on any PC
running windows, where it can be run from the Windows command prompt.

\subsubsection*{Preparing the build environment}

You need to install the following packages
\begin{itemize}
\item MinGW 
\item MSYS  
\item NSIS  
\end{itemize}

MinGW (Minimalist GNU for Windows) is a build environment for Windows.
There is an official release from \url{http://www.mingw.org}.
However, we used the MinGW distribution that comes as part of the R
tools for windows (\url{http://www.murdoch-sutherland.com/Rtools}),
since the compilers in this distribution match the compilers used to
build the binary distribution of R for windows, as well as the R
packages distributed via CRAN (\url{http://cran.r-project.org}).

We used version 2.7 of \verb+Rtools+.  You only need to install the
``MingGW components and tools'', not the other components of
\verb+Rtools+. The installer will also ask if you wish to modify the
Windows PATH. You do not need this.

MSYS (the Minimal SYStem) provides a bash shell for you to build Unix
software. These instructions were tested with MSYS 1.0.10.  It can be
downloaded, as a self-extracting executable, from the MinGW web site
\url{http://www.mingw.org}. At the end of the installation process, it
will launch a post-install script that will allow you to use MSYS in
conjunction with MinGW. Note that you do {\em not} need the MSYS
developer toolkit (DTK) to compile \JAGS.

MSYS creates a home directory for you in
\verb+c:\msys\<version>\home\<username>+, where \texttt{<version>} is
the version of MSYS and \texttt{<username>} is your user name under
Windows. You will need to copy and paste the source files for LAPACK
and JAGS into this directory.

The Nullsoft Scriptable Install System
(\url{http:\\nsis.sourceforge.net}) allows you to create a
self-extracting executable that installs \JAGS\ on the target PC.
These instructions were tested with NSIS 2.33.

\subsubsection*{LAPACK}

Download the LAPACK source file \verb+lapack-lite-3.1.1.tgz+ from
\url{http://www.netlib.org/lapack} and unpack it in your home
directory.
\begin{verbatim}
tar xfvz lapack-lite-3.1.1.tgz
cd lapack-lite-3.1.1
\end{verbatim}
Replace the file \texttt{make.inc} with
\texttt{INSTALL/make.inc.LINUX}.  Then edit \texttt{make.inc},
replacing the line
\begin{verbatim}
PLAT = _LINUX
\end{verbatim}
with something more sensible, like
\begin{verbatim}
PLAT = _MinGW
\end{verbatim} 
If using gcc 4.2 (the version that comes with Rtools 2.7), you need
to set the \verb+FORTRAN+ and \verb+LOADER+ variables so that they use
gfortran.
\begin{verbatim}
FORTRAN = gfortran
LOADER = gfortran
\end{verbatim}
and uncomment the line
\begin{verbatim}
TIME = INT_ETIME
\end{verbatim}

Edit the file \texttt{Makefile} so that it builds the BLAS library. The
line that starts \texttt{lib:} should read
\begin{verbatim}
lib: blaslib lapacklib tmglib
\end{verbatim}
Type
\begin{verbatim}
make 
\end{verbatim}
The compilation process is slow. Eventually, it will create two static
libraries \verb+blas_MinGW.a+ and \verb+lapack_MingGW.a+. These are
insufficient for building \JAGS: you need to create dynamic link
library (DLL) for each one.

First create a definition file \verb+blas.def+ that exports all the
symbols from the BLAS library
\begin{verbatim}
dlltool -z blas.def --export-all-symbols blas_MinGW.a
\end{verbatim}
Then link this with the static library to create a DLL
(\verb+blas.dll+) and an import library (\verb+libblas.dll.a+)
\begin{verbatim}
gcc -shared -o blas.dll -Wl,--out-implib=libblas.dll.a \
blas.def blas_MinGW.a -lgfortran
\end{verbatim}
(If using gcc 3.4, the library should be linked with \verb+-lg2c+ 
instead of \verb+-lgfortran+)
 
Repeat the same steps for the LAPACK library, creating an import library
(\verb+liblapack.dll.a+) and DLL (\verb+lapack.dll+)
\begin{verbatim}
dlltool -z lapack.def --export-all-symbols lapack_MinGW.a
gcc -shared -o lapack.dll -Wl,--out-implib=liblapack.dll.a \
lapack.def lapack_MinGW.a  -L./ -lblas -lgfortran
\end{verbatim}

\subsubsection{Compiling \JAGS}

Unpack the JAGS source
\begin{verbatim}
tar xfvz JAGS-1.0.3.tar.gz
cd JAGS-1.0.3
\end{verbatim}
and configure JAGS
\begin{verbatim}
./configure LDFLAGS="-L/path/to/import/libs/" 
\end{verbatim}
where \verb+/path/to/import/libs+ is a directory that contains the
import libraries (\verb+libblas.dll.a+ and \verb+liblapack.dll.a+).
This must be an {\em absolute} path name, and not relative to
the JAGS build directory.

Normally you will want to distribute the blas and lapack libraries
with JAGS.  In this case, put the DLLs and import libraries in the
sub-directory \verb+win32/lapack+. They will be detected and included
with the distribution.

Make sure that the file \verb+makensis.exe+, provided by NSIS, is in
your PATH. For a typical installation of NSIS:
\begin{verbatim}
PATH=$PATH:/c/Program\ files/NSIS
\end{verbatim}
Then type
\begin{verbatim}
make win32-installer
\end{verbatim}
The self extracting archive will be in the subdirectory \verb+win32+.

Note that you must go straight from the configure step to \texttt{make
  win32-installer} without the usual step of typing \texttt{make} on
its own.  The \texttt{win32-installer} make target resets the
installation prefix, and this will cause an error if the source is
already compiled.

\end{document}



