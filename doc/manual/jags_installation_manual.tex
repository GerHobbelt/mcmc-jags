\documentclass[11pt, a4paper, titlepage]{article}
\usepackage{amsmath}
\usepackage{a4wide}
\usepackage{url}
\usepackage{multirow}
\usepackage{amsfonts}
\newcommand{\release}{2.2.0}
\newcommand{\JAGS}{\textsf{JAGS}}
\newcommand{\BUGS}{\textsf{BUGS}}
\newcommand{\WinBUGS}{\textsf{WinBUGS}}
\newcommand{\R}{\textsf{R}}
\newcommand{\CODA}{\textsf{coda}}
\begin{document}

\title{JAGS Version \release\ installation manual}
\author{Martyn Plummer \and Bill Northcott}
\maketitle

\JAGS\ is distributed in binary format for Microsoft Windows, Mac OS
X, and most Linux distributions.  The following instructions are for
those who wish to build \JAGS\ from source. The manual is divided
into three sections with instructions for Linux/Unix, Mac OS X, and Windows.

\section{Linux and UNIX}

\JAGS\ follows the usual GNU convention of 
\begin{verbatim}
./configure
make
make install
\end{verbatim}
which is described in more detail in the file \texttt{INSTALL} in
the top-level source directory. On some UNIX platforms, you may
be required to use GNU make (gmake) instead of the native make
command. On systems with multiple processors, you may use the option 
\verb+-j+ to speed up compilation, {\em e.g.} for a quad-core PC you
may use:
\begin{verbatim}
make -j4
\end{verbatim}

\subsection{Configure options}

At configure time you also have the option of defining options such
as:
\begin{itemize}
\item The names of the C, C++, and Fortran compilers.  Although
  \JAGS\ contains no Fortran code, you are required to define a
  Fortran compiler so that \JAGS\ modules can be linked against
  libraries written in Fortran (such as BLAS and LAPACK)
\item Optimization flags for the C and C++ compilers.  \JAGS\ is
  optimized by default if the GNU compiler (gcc) is used. Otherwise
  you must explicitly supply optimization flags.
\item Installation directories. \JAGS\ conforms to the GNU standards
  for where files are installed. You can control the installation
  directories in more detail using the flags that are listed when
  you type \verb+./configure --help+.
\end{itemize}

\subsubsection{Configuration for a 64-bit build}

By default, JAGS will install all libraries into
\verb+/usr/local/lib+.  If you are building a 64-bit version of \JAGS,
this may not be appropriate for your system. On Fedora and other
RPM-based distributions, for example, 64-bit libraries should be
installed in \verb+lib64+, and on Solaris, 64-bit libraries are in a
subdirectory of \verb+lib+ ({\em e.g.} \verb+lib/amd64+ if you are
using a x86-64 processor), whereas on Debian, and other Linux
distributions that conform to the FHS, the correct installation
directory is \verb+lib+.

To ensure that \JAGS\ libraries are installed in the correct
directory, you should supply the \verb+--libidr+ argument to the
configure script, {\em e.g.}:
\begin{verbatim}
./configure --libdir=/usr/local/lib64
\end{verbatim}

It is important to get the installation directory right when using the
\texttt{rjags} interface between R and \JAGS, otherwise the
\texttt{rjags} package will not be able to find the \JAGS\ library.

\subsubsection{Configuration for a private installation}

If you do not have administrative privileges, you may wish to install
\JAGS\ in your home directory. This can be done with the following
configuration options
\begin{verbatim}
export JAGS_HOME=$HOME/jags #or wherever you want it
./configure --bindir=$JAGS_HOME/bin --libdir=$JAGS_HOME/lib \
 --libexecdir=$JAGS_HOME/bin --includedir=$JAGS_HOME/include
\end{verbatim}
You then need to modify your PATH environment variable to include
\verb+$JAGS_HOME/bin+. You may also need to set \verb+LD_LIBRARY_PATH+
to include \verb+$JAGS_HOME/lib+ (On Linux this is not necessary as
the location of libjags and libjrmath is hard-coded into the
\JAGS\ binary).

\subsection{BLAS and LAPACK}
\label{section:blas:lapack}

BLAS (Basic Linear Algebra System) and LAPACK (Linear Algebra Pack)
are two libraries of routines for linear algebra. They are used by the
multivariate functions and distributions in the \texttt{bugs} module.
Most unix-like operating system vendors supply shared libraries that
provide the BLAS and LAPACK functions, although the libraries may not
literally be called ``blas'' and ``lapack''.  During configuration, a
default list of these libraries will be checked. If \texttt{configure}
cannot find a suitable library, it will stop with an error message.

You may use alternative BLAS and LAPACK libraries using the configure
options \texttt{--with-blas} and \texttt{--with-lapack}
\begin{verbatim}
./configure --with-blas="-lmyblas" --with-lapack="-lmylapack"
\end{verbatim}

If the BLAS and LAPACK libraries are in a directory that is not on the
default linker path, you must set the \verb+LDFLAGS+ environment variable
to point to this directory at configure time:
\begin{verbatim}
LDFLAGS="-L/path/to/my/libs" ./configure ...
\end{verbatim}

At runtime, if you have linked \JAGS\ against BLAS or LAPACK in
a non-standard location, you must supply this location with the
environment variable \verb+LD_LIBRARY_PATH+, {\em e.g.}
\begin{verbatim}
LD_LIBRARY_PATH="/path/to/my/libs:${LD_LIBRARY_PATH}"
\end{verbatim} %$
Alternatively, you may hard-code the paths to the blas and lapack
libraries at compile time. This is compiler and platform-specific,
but is typically achieved with
\begin{verbatim}
LDFLAGS="-L/path/to/my/libs -R/path/to/my/libs
\end{verbatim}

\subsubsection{Multithreaded BLAS and LAPACK}
\label{section:blas:multithreaded}

Some high-performance computing libraries offer multi-threaded
versions of the BLAS and LAPACK libraries. Although instructions for
linking against some of these libraries are given below, this should
not be taken as encouragement to use multithreaded BLAS.  Testing
shows that using multiple threads in BLAS can lead to significantly
{\em worse} performance while using up substantially more computing
resources.

\subsection{GNU/Linux}
\label{section:gnulinux}

GNU/Linux is the development platform for \JAGS, and a variety of
different build options have been explored, including the use of
third-party compilers and linear algebra libraries.

\subsubsection{Fortran compiler}

The GNU FORTRAN compiler changed between gcc 3.x and gcc 4.x from
\verb+g77+ to \verb+gfortran+. Code produced by the two compilers is
binary incompatible. If your BLAS and LAPACK libraries are linked
against \verb+libgfortran+, then they were built with \verb+gfortran+
and you must also use this to compile \JAGS. 

Most recent GNU/Linux distributions have moved completely to gcc 4.x.
However, some older systems may have both compilers installed.
Unfortunately, if \verb+g77+ is on your path then the configure script
will find it first, and will attempt to use it to build \JAGS. This
results in a failure to recognize the installed BLAS and LAPACK
libraries. In this event, set the \verb+F77+ variable at configure time.
\begin{verbatim}
F77=gfortran ./configure
\end{verbatim}

\subsubsection{BLAS and LAPACK}

The {\bf BLAS} and {\bf LAPACK} libraries from Netlib
(\url{http://www.netlib.org}) should be provided as part of your Linux
distribution. If your Linux distribution splits packages into ``user''
and ``developer'' versions, then you must install the developer
package ({\em e.g.}  \texttt{blas-devel} and \texttt{lapack-devel}).

{\bf Suse Linux Enterprise Server (SLES)} does not include BLAS and
LAPACK in the main distribution. They are included in the SLES SDK, on
a set of CD/DVD images which can be downloaded from the Novell web
site.  See \url{http://developer.novell.com/wiki/index.php/SLES_SDK}
for more information.

It is quite common for the Netlib implementations of BLAS and LAPACK
to break when they are compiled with the latest GNU compilers.  Linux
distributions that use ``bleeding edge'' development tools -- such as
{\bf Fedora} -- may ship with a broken version of BLAS and
LAPACK. Normally, this problem is quickly identified and
fixed. However, you need to take care to use the online updates of the
BLAS and LAPACK packages from your Linux Distributor, and not rely on
the version that came on the installation disk.

\subsubsection{ATLAS}

On Fedora Linux, pre-compiled atlas libraries are available via the
\texttt{atlas} and \texttt{atlas-devel} RPMs.  These RPMs install the
atlas libraries in the non-standard directory \texttt{/usr/lib/atlas}
(or \texttt{/usr/lib64/atlas} for 64-bit builds) to avoid conflicts
with the standard \texttt{blas} and \texttt{lapack} RPMs. To use the
atlas libraries, you must supply their location using the
\verb+LDFLAGS+ variable (see section \ref{section:blas:lapack})
\begin{verbatim}
./configure LDFLAGS="-L/usr/lib/atlas"
\end{verbatim}
Runtime linking to the correct libraries is ensured by the automatic
addition of \texttt{/usr/lib/atlas} to the linker path (see the file
\texttt{/etc/ld.so.conf}), so you do not need to set the
environment variable \verb+LD_LIBRARY_PATH+ at run time.

\subsubsection{AMD Core Math Library}
\label{section:acml:linux}

The AMD Core Math Library (acml) provides optimized BLAS and LAPACK
routines for AMD processors. To link \JAGS\ with \texttt{acml}, you must
supply the \texttt{acml} library as the argument to \texttt{--with-blas}.
It is not necessary to set the \texttt{--with-lapack} argument
as \texttt{acml} provides both sets of functions. See also
section~\ref{section:blas:lapack} for run-time instructions.

For example, to link to the 64-bit acml using gcc 4.0+:
\begin{verbatim}
LDFLAGS="-L/opt/acml4.3.0/gfortran64/lib" \
./configure --with-blas="-lacml -lacml_mv" 
\end{verbatim}
The library \verb+acmv_mv+ library is a vectorized math library that
exists only for the 64-bit version and is omitted when linking against
32-bit acml.

On multi-core systems, you may wish to use the threaded acml library
(See the warning in section \ref{section:blas:multithreaded} however).
To do this, link to \verb+acml_mp+ and add the compiler flag
\verb+-fopenmp+:
\begin{verbatim}
LDFLAGS="-L/opt/acml4.3.0/gfortran64_mp/lib" \
CXXFLAGS="-O2 -g -fopenmp" ./configure --with-blas="-lacml_mp -lacml_mv" 
\end{verbatim}
The number of threads used by multi-threaded acml may be controlled
with the environment variable \verb+OMP_NUM_THREADS+.

\subsubsection{Intel Math Kernel Library}

The Intel Math Kernel library (MKL) provides optimized BLAS and LAPACK
routines for Intel processors. The instructions below are for MKL
version 10.0 and above which use a ``pure layered'' model for linking.
The layered model gives the user fine-grained control over four
different library layers: interface, threading, computation, and
run-time library support. Some examples of linking to MKL using this
layered model are given below. These examples are for GCC compilers on
\verb+x86_64+. The choice of interface layer is important on
\verb+x86_64+ since the Intel Fortran compiler returns complex values
differently from the GNU Fortran compiler. You must therefore use the
interface layer that matches your compiler (\verb+mkl_intel*+ or
\verb+mkl_gf*+).
For further guidance, consult the MKL Link Line advisor at
\url{http://software.intel.com/en-us/articles/intel-mkl-link-line-advisor}.

I have not been able to link \JAGS\ with MKL using GNU compilers, except
which building a static version. To build a static version of \JAGS,
use the configure option \verb+--disable-shared+. The \JAGS\ library and
modules will be linked into the main executable.

JAGS can be linked to a sequential version of MKL by
\begin{verbatim}
MKL_HOME=/opt/intel/mkl/10.0.3.020/
MKL_LIB_PATH=${MKL_HOME}/lib/em64t/
./configure --disable-shared \
            --with-blas="-L${MKL_LIB_PATH} -lmkl_gf_lp64 -lmkl_sequential -lmkl_core"
\end{verbatim}

Threaded MKL may be used with:
\begin{verbatim}
./configure --disable-shared\
   --with-blas="-L${MKL_LIB_PATH} -lmkl_gf_lp64 -lmkl_gnu_thread -lmkl_core -liomp5 -lpthread"
\end{verbatim}
The default number of threads will be chosen by the OpenMP software,
but can be controlled by setting \verb+OMP_NUM_THREADS+ or \verb+MKL_NUM_THREADS+.  (See the warning in section \ref{section:blas:multithreaded} however).

\subsubsection{Using Intel Compilers}

\JAGS\ has been successfully built with the Intel C, C++ and Fortran
compilers.  The additional configure options required to use the Intel
compilers are:
\begin{verbatim}
source /opt/intel/Compiler/11.1/bin/ifortvars.sh
source /opt/intel/Compiler/11.1/bin/iccvars.sh
CC=icc CXX=icpc F77=ifort ./configure 
\end{verbatim}

\subsection{OpenSolaris}

\JAGS\ has been successfully built and tested on the Intel x86
platform under Solaris 11 Express using the Sun Studio Express 6/10
compilers.
\begin{verbatim}
./configure CC=cc CXX=CC F77=f95 \
CFLAGS="-xO3 -xarch=sse2" \
FFLAGS="-xO3 -xarch=sse2" \
CXXFLAGS="-xO3 -xarch=sse2"
\end{verbatim}
The Sun Studio compiler is not optimized by default. Use the option
\verb+-xO3+ for optimization (NB This is the letter ``O'' not the
number 0) In order to use the optimization flag \verb+-xO3+ you
must specify the architecture with the \verb+-xarch+ flag. The options
above are for an Intel processor with SSE2 instructions. This must be
adapted to your own platform.

To compile a 64-bit version of JAGS, add the option \verb+-m64+ to
all the compiler flags.

Solaris provides two versions of the C++ standard library:
\texttt{libCstd}, which is the default, and \texttt{libstlport4},
which conforms more closely to the C++ standard. \JAGS\ may be linked
to the stlport4 library by adding the options
\verb+-library=stlport4+ to \verb+CXXFLAGS+. 

The configure script automatically detects the Sun Performance library,
which implements the BLAS/LAPACK functions.  Automatic detection may
not work on older versions of Sun Studio, which used a different syntax
for specifying this library.  In this case, you may need to use the
configure option
\begin{verbatim}
--with-blas="-xlic_lib=sunperf -lsunmath"
\end{verbatim}

\subsection{IRIX}

\JAGS\ has not been tested on IRIX for some time.  Version 1.0.0 was
successfully built using the MIPSpro 7.4 compiler on IRIX 6.5. The
following configure options were used:
\begin{verbatim}
./configure CC=cc CXX=CC F77=f77 \
CFLAGS="-O2 -g2 -OPT:IEEE_NaN_inf=ON" \
CXXFLAGS="-O2 -g2 -OPT:IEEE_NaN_inf=ON" 
\end{verbatim}
and \JAGS\ was built with \verb+gmake+ (GNU make).

BLAS and LAPACK functions on IRIX are provided by the Scientific
Computing Software library (\verb+scs+). The presence of this library
is detected automatically by the configure script.

When using the MIPSpro compiler, optimization flags must be given
explicitly at configure time. If this is not done, then \JAGS\ will
not be optimized at all and will run slowly.

\clearpage
\section{Mac OS X}

If trying to build software on Mac OS X you really need to use Leopard
(10.5.x) or Snow Leopard (10.6.x). Unless otherwise stated these
instruction assume Snow Leopard (10.6.x). The open source support has
improved greatly in recent releases. You also need the latest version
of Apple's Xcode development tools. The current version is Xcode 3.2.x
(Leopard uses 3.1.x).  Early versions have serious bugs which affect R
and \JAGS.  Xcode is available as a free download from
\url{http://developer.apple.com}. You need to set up a free login to
ADC. The Apple developer tools do not include a Fortran
compiler. Without Fortran, you will not be able to build \JAGS.

For instructions for building on Tiger or for older versions of
\R\ see previous versions of this manual.

The GNU gfortran Fortran compiler is included in the \R\ binary
distribution available on CRAN. Install the \R\ binary and select all
the optional components in the `Customize' step of the installer.
These instructions assume R-2.7.x.

The default C/C++ compiler for Snow Leopard is gcc-4.2.x. Xcode 3.2
also includes gcc-4.2 and llvm-gcc4.2.  The code has been successfully
built with these optional compilers but will only run on Leopard.
llvm is being actively developed by Apple and may produce better code.

MacOS X 10.2 and onwards include optimised versions of the BLAS and
LAPACK libraries.  So no extra libraries are is needed for Snow
Leopard.  Optimisation continues andApple are working on using GPUs
for this sort of math.  Make sure your OS is up to date.

To ensure the \JAGS\ configure script can find the Fortran compiler
for a bash shell
\begin{verbatim}
export F77=/usr/local/bin/gfortran
\end{verbatim}

On 64 bit hardware, which means most recent Macs, there may be a 
problem with the Fortran compiler.  Apple's compilers default to 64 bit 
builds on 64 bit hardware but the Fortran binaries available default to 
32 bit builds. This means you need to add compile and link options.

For instance on 64 bit Intel Macs type
\begin{verbatim}
export CFLAGS='-arch x86_64'
export CXXFLAGS='-arch x86_64'
export FFLAGS='-arch x86_64'
export LDFLAGS='-arch x86_64'
\end{verbatim}

Some Fortran compilers (not the ones from CRAN) do not understand 
the -arch option. For these you will need something like:
\begin{verbatim}
export CFLAGS='-arch x86_64'
export CXXFLAGS='-arch x86_64'
export FFLAGS='-m64'
export LDFLAGS='-arch x86_64'
\end{verbatim}

To build \JAGS\ unpack the source code and cd into the source  
directory. Type the following:
\begin{verbatim}
./configure
make
\end{verbatim}
(if you have multiple CPUs try \verb+make -j 4+ or
\verb+make -j 8+. It may need to be issued more than once)
\begin{verbatim}
sudo make install
\end{verbatim}

You need to ensure \texttt{/usr/local/bin} is in your PATH in order
for `jags' to work from a shell prompt.

This will build the default architecture for you Mac: ppc on a G4 or
G5 and i386 or \verb+x86_64+ on an Intel Mac.  If you want to build
multiple architecture fat binaries, you will need to ensure that
libtool in the JAGS sources is version 1.5.24 or later.  Then you can
use configure commands like
\begin{verbatim}
CXXFLAGS="-arch i386 -arch x86_64" ./configure
\end{verbatim}

Make will then build fat binaries.  See the R Mac developers page 
\url{http://r.research.att.com/} for instructions to build fat R packages.

A final note on MacOS X builds: do NOT use \texttt{-O3}.  It is not optimal 
and may find compiler bugs.  Apple recommends \texttt{-Os}.

\clearpage
\section{Windows}
\label{section:windows}

These instructions use MinGW, the Minimalist GNU system for Windows.
You need some familiarity with Unix in order to follow the build
instructions but, once built, \JAGS\ can be installed on any PC
running windows, where it can be run from the Windows command prompt.

\subsection{Preparing the build environment}

You need to install the following packages
\begin{itemize}
\item The TDM-GCC compiler suite for Windows
\item MSYS  
\item NSIS, including the AccessControl plug-in  
\end{itemize}

MinGW (Minimalist GNU for Windows) is a build environment for Windows.
There is an official release from \url{http://www.mingw.org}.
However, we used the TDM-GCC distribution
(\url{http://tdm-gcc-tdragon.net}).  This distribution was chosen
because it allows us to build a version of JAGS that is statically
linked against the gcc runtime library.  This, in turn, is necessary
to have a functional rjags package on Windows.  We also tried version
213 of Rtools
(\url{http://www.murdoch-sutherland.com/Rtools}). Although the
resulting JAGS library is functional, it is not compatible with R:
loading the rjags package causes R to crash on exit.

TDM-GCC has a nice installer, available from sourceforge (follow the
links on the main TDM-GCC web site). Select a ``Recommended C/C++''
installation and customize it by selecting the fortran compiler, which
is not installed by default. After installation, to force the compiler
to use static linking, delete any import libraries (files ending in
\verb+.dll.a+) in the TDM-GCC tree.  The 32-bit and 64-bit versions
of TDM-GCC are installed separately. You need both to make the JAGS
installer.

MSYS (the Minimal SYStem) is part of the MinGW project. It provides a
bash shell for you to build Unix software. These instructions were
tested with MSYS 1.0.11, the last version of MSYS to be bundled with a
Windows installer.  The installer can be downloaded from
\url{http://sourceforge.net/projects/mingw/files}.  

MSYS creates a home directory for you in
\verb+c:\msys\<version>\home\<username>+, where \texttt{<version>} is
the version of MSYS and \texttt{<username>} is your user name under
Windows. You will need to copy and paste the source files for LAPACK
and JAGS into this directory.

The Nullsoft Scriptable Install System
(\url{http://nsis.sourceforge.net}) allows you to create a
self-extracting executable that installs \JAGS\ on the target PC.
These instructions were tested with NSIS 2.46.  You must also install
the AccessControl plug-in for NSIS, which is available from
\url{http://nsis.sourceforge.net/AccessControl_plug-in}.

\subsubsection{Building LAPACK}

Download the LAPACK source file from
\url{http://www.netlib.org/lapack}. We used version 3.3.0.  Unpack the
file in your home directory.
\begin{verbatim}
tar xfvz lapack-3.3.0.tgz
cd lapack-3.3.0
\end{verbatim}
Copy the file \texttt{INSTALL/make.inc.gfortran} to \texttt{make.inc} in
the top level source directory.  Then edit \texttt{make.inc},
replacing the line
\begin{verbatim}
PLAT = _LINUX
\end{verbatim}
with something more sensible, like
\begin{verbatim}
PLAT = _MinGW
\end{verbatim} 

Edit the file \texttt{Makefile} so that it builds the BLAS library. The
line that starts \texttt{lib:} should read
\begin{verbatim}
lib: blaslib lapacklib tmglib
\end{verbatim}
Type
\begin{verbatim}
make 
\end{verbatim}
The compilation process is slow. Eventually, it will create two static
libraries \verb+blas_MinGW.a+ and \verb+lapack_MingGW.a+. These are
insufficient for building \JAGS: you need to create dynamic link
library (DLL) for each one.

First create a definition file \verb+libblas.def+ that exports all the
symbols from the BLAS library
\begin{verbatim}
dlltool -z libblas.def --export-all-symbols blas_MinGW.a
\end{verbatim}
Then link this with the static library to create a DLL
(\verb+libblas.dll+) and an import library (\verb+libblas.dll.a+)
\begin{verbatim}
gcc -shared -o libblas.dll -Wl,--out-implib=libblas.dll.a \
libblas.def blas_MinGW.a -lgfortran
\end{verbatim}
 
Repeat the same steps for the LAPACK library, creating an import library
(\verb+liblapack.dll.a+) and DLL (\verb+liblapack.dll+)
\begin{verbatim}
dlltool -z liblapack.def --export-all-symbols lapack_MinGW.a
gcc -shared -o liblapack.dll -Wl,--out-implib=liblapack.dll.a \
liblapack.def lapack_MinGW.a  -L./ -lblas -lgfortran
\end{verbatim}

Repeat these instructions for both 32-bit and 64-bit Windows using the
respective versions of TDM-GCC.

Note that, although the 64-bit version of TDM-GCC is a cross-compiler,
you cannot cross-build 64-bit BLAS and LAPACK on a 32-bit Windows
system. This is because the build process includes compilation of test
programs which must be run.

\subsection{Compiling \JAGS}

Unpack the JAGS source
\begin{verbatim}
tar xfvz JAGS-3.0.0.tar.gz
cd JAGS-3.0.0
\end{verbatim}
and configure JAGS
\begin{verbatim}
./configure LDFLAGS="-L/path/to/import/libs/ -Wl,--enable-auto-import" 
\end{verbatim}
where \verb+/path/to/import/libs+ is a directory that contains the
import libraries (\verb+libblas.dll.a+ and \verb+liblapack.dll.a+).
This must be an {\em absolute} path name, and not relative to
the JAGS build directory.

If you are cross-compiling a 64-bit version of JAGS on 32-bit windows,
you need to add the configure flag \verb+--host=x86_64-win64-mingw32+.

Normally you will want to distribute the blas and lapack libraries
with JAGS.  In this case, put the 32-bit DLLs and import libraries in
the sub-directory \verb+win/runtime32+ and the 64-bit DLLs and import
libraries in the sub-directory \verb+win/runtime64+. They will be
detected and included with the distribution.

To install the 32-bit version of JAGS, after the configure step, type
\begin{verbatim}
make win32-installer
\end{verbatim}
This will install JAGS into the subdirectory \verb+win/inst32+.
Note that you must go straight from the configure step to \texttt{make
  win32-install} without the usual step of typing \texttt{make} on
its own.  The \texttt{win32-install} target resets the installation
prefix, and this will cause an error if the source is already
compiled.

To install the 64-bit version, \verb+make clean+, reconfigure JAGS
and type 
\begin{verbatim}
make win64-installer
\end{verbatim}
This will install JAGS into the subdirectory \verb+win/inst64+.

With both 32-bit and 64-bit installations in place you can create the
installer. Make sure that the file \verb+makensis.exe+, provided by
NSIS, is in your PATH. For a typical installation of NSIS, on 32-bit
windows:
\begin{verbatim}
PATH=$PATH:/c/Program\ files/NSIS
\end{verbatim}
Then type
\begin{verbatim}
make installer
\end{verbatim}
After the build process finishes, the self extracting archive will be
in the subdirectory \verb+win+.


To build 64-bit \JAGS\, just add the configure option
\begin{verbatim}
--host=x86_64-w64-mingw32
\end{verbatim}

Then, to build the NSIS installer, you should use the make
target \verb+win64-installer+ instead of \verb+win32-installer+.

\end{document}

